% gecheckt mit chat-gpt
\begin{boxK}{Zweifaktorielle Varianzanalyse}
\begin{center}
 
Hinweis: gilt nur bei balancierten Designs (gleiche Stichprobengröße)
\end{center}

\begin{multicols}{2}
\subsection*{F-Werte} 
$$F_A=\frac{\hat{\sigma}_A^2}{\hat{\sigma}^2_{\mathrm{inn}}}$$ $$F_B=\frac{\hat{\sigma}_B^2}{\hat{\sigma}^2_{\mathrm{inn}}}$$ $$F_{A\times B}=\frac{\hat{\sigma}_{A\times B}^2}{\hat{\sigma}^2_{\mathrm{inn}}}$$

\subsection*{Populationsvarianzen}
$$\hat{\sigma}_A^2=\frac{\mathrm{QS}_{A}}{\mathrm{df}_{A}}$$ $$\hat{\sigma}_B^2=\frac{\mathrm{QS}_{B}}{\mathrm{df}_{B}}$$  $$\hat{\sigma}_{A\times B}^2=\frac{\mathrm{QS}_{A\times B}}{\mathrm{df}_{A\times B}}$$
$$\hat{\sigma}^2_ {\mathrm{inn}}=\frac{\mathrm{QS}_{\mathrm{inn}}}{\mathrm{df}_{\mathrm{inn}}}$$

\subsection*{Freiheitsgrade}
\begin{align*}
%\mathrm{df}_{\mathrm{ges}}&=N-1\\
%\mathrm{df}_{\mathrm{inn}}&=k\cdot m (n-1)\\
\mathrm{df}_{\mathrm{inn}}&=\sum\limits_j^k \sum\limits_l^m (n_{jl}-1) = N-km \\
\mathrm{df}_{\mathrm{A}}&=k-1\\
\mathrm{df}_{\mathrm{B}}&=m-1\\
\mathrm{df}_{\mathrm{AxB}}&=(k-1)(m-1)\\
\end{align*}

\subsection*{Quadratsummen}
\begin{align*}
\mathrm{QS}_{\mathrm{ges}}&=\mathrm{QS}_A+\mathrm{QS}_B+ \mathrm{QS}_{A\times B}+\mathrm{QS}_\mathrm{inn}\\
\mathrm{QS}_{\mathrm{ges}}&=\sum\limits_j^k \sum\limits_l^m \sum\limits_i^n (x_{ijl}-\bar{\bar{x}})^2\\
\mathrm{QS}_{\mathrm{inn}}&=\sum\limits_j^k \sum\limits_l^m \sum\limits_i^n(x_{ijl}-\bar{x}_{jl})^2\\
\mathrm{QS}_A&=n \cdot m \sum\limits_j^k (\bar{x}_{j\cdot}-\bar{\bar{x}})^2\\
\mathrm{QS}_B&= n \cdot k \sum\limits_l^m (\bar{x}_{\cdot l}-\bar{\bar{x}})^2\\
\mathrm{QS}_{A\times B}&= n \sum\limits_{j=1}^k \sum\limits_{l=1}^m \left( \bar{x}_{jl} - \bar{x}_{j\cdot} - \bar{x}_{\cdot l} + \bar{\bar{x}} \right)^2 \\
\mathrm{QS}_{A\times B}&=\mathrm{QS}_{\mathrm{ges}}-\mathrm{QS}_{\mathrm{A}}-\mathrm{QS}_{\mathrm{B}}-\mathrm{QS}_{\mathrm{inn}}\\
\end{align*}

\subsection*{Notation}
\begin{tabular}{rl}
$N$: &Gesamtzahl der Messwerte der Untersuchung \\
$n$: &Anzahl der Messwerte pro Bedingung \\
$k$: &Anzahl der Stufen des Faktors $A$\\
$m$: &Anzahl der Stufen des Faktors $B$\\
$\bar{\bar{x}}$: &Gesamtmittelwert\\
$\bar{x}_{j\cdot}$: &Mittelwert für Faktor $A$, Bedingung $j$\\
$\bar{x}_{\cdot l}$: &Mittelwert für Faktor $B$, Bedingung $l$\\
\end{tabular}
\end{multicols}


\end{boxK}
