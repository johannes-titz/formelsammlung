% !TeX program = lualatex
% github pdf link: https://johannes-titz.github.io/formelsammlung/main.pdf 

\documentclass[9pt,a4paper]{article}
\usepackage{csquotes}
\usepackage{icomma}
%\usepackage{fontawesome5}
%\usepackage{caption}
%\usepackage{float}
%\usepackage{booktabs}
\usepackage{amsmath}
%\usepackage[utf8]{inputenc}
\usepackage[ngerman]{babel}
%\usepackage[T1]{fontenc}
\usepackage{fancyhdr}
\usepackage{tcolorbox}
\usepackage{booktabs}
\usepackage{apacite}
\usepackage{hang}
\tcbuselibrary{skins} %??

\usepackage[colorlinks]{hyperref}

\def\tmp#1#2#3{%
  \definecolor{Hy#1color}{#2}{#3}%
  \hypersetup{#1color=Hy#1color}}
\tmp{link}{HTML}{800006}
\tmp{cite}{HTML}{2E7E2A}
\tmp{file}{HTML}{131877}
\tmp{url} {HTML}{8A0087}
\tmp{menu}{HTML}{727500}
\tmp{run} {HTML}{137776}
\def\tmp#1#2{%
  \colorlet{Hy#1bordercolor}{Hy#1color#2}%
  \hypersetup{#1bordercolor=Hy#1bordercolor}}
\tmp{link}{!60!white}
\tmp{cite}{!60!white}
\tmp{file}{!60!white}
\tmp{url} {!60!white}
\tmp{menu}{!60!white}
\tmp{run} {!60!white}
\usepackage{tikz}
\usetikzlibrary{shapes,positioning,arrows,fit,calc,graphs,graphs.standard}
\usepackage{multicol}
\usepackage{wrapfig}
\usepackage[top=10mm,bottom=15mm,left=10mm,right=10mm]{geometry}
%\usepackage[framemethod=tikz]{mdframed}
\usepackage{microtype}
\setlength{\arrayrulewidth}{0.5mm}
%\setlength{\tabcolsep}{8pt}
\renewcommand{\arraystretch}{1.25}

\usepackage{inputenc}

%\pagestyle{fancy}
%\fancyhf{}
%\fancyhead[R]{}
%\fancyhead[L]{}
%\fancyfoot[C]{}%\thepage}
%\thispagestyle{empty}
% remove page numbers
%\pagenumbering{gobble}

%\fancypagestyle{plain}{%  the preset of fancyhdr %
%    \fancyhf{} % clear all header and footer %fields
%}
%    \fancyfoot[C]{\textbf{\thepage}} % except  %the center
%    \renewcommand{\headrulewidth}{0pt}
%    \renewcommand{\footrulewidth}{0pt}}
    
\usepackage[defaultlines=10,all]{nowidow}
%\usepackage[sc,osf]{mathpazo}   % With old-style figures and real smallcaps.
%\linespread{1.025}              % Palatino leads a little more leading

% Euler for math and numbers
%\usepackage[euler-digits,small]{eulervm}
\AtBeginDocument{\renewcommand{\hbar}{\hslash}}
%\widowpenalties 1 10000
%\raggedbottom
%\interlinepenalty 10000


\let\bar\overline
\raggedcolumns

%\usepackage{placeins} % put this in your pre-amble
%\usepackage{flafter}  % put this in your pre-amble

\makeatletter
\renewcommand{\section}{\@startsection{section}{1}{0mm}%
                                {.6ex}%
                                {0.25ex}%x
                                {\large\bfseries}}
\renewcommand{\subsection}{\@startsection{subsection}{2}{0mm}%
                                {2ex}%
                                {0.5ex}%x
                                {\normalsize\bfseries}}
\renewcommand{\subsubsection}{\@startsection{subsection}{3}{0mm}%
                                {.2ex}%
                                {0.2ex}%x
                                {\footnotesize\bfseries\raggedleft}}
\makeatother
\setlength{\parskip}{0.5\baselineskip}%
\setlength{\parindent}{0pt}
\setlength{\baselineskip}{0.5pt}

\setlength\columnsep{15pt} % This is the default columnsep for all pages

\newtcolorbox{boxK}[1]{
    %before skip=10pt, after skip=10pt,
    sharpish corners, % better drop shadow
    colback = white,
    boxrule = 0pt,
    toprule = 4.5pt, % top rule weight
    bottom = 20pt,
    enhanced,
    fuzzy shadow = {0pt}{-2pt}{-0.5pt}{0.5pt}{black!35}, % {xshift}{yshift}{offset}{step}{options}
    title=#1,
    fonttitle=\bfseries,
    attach boxed title to top center={yshift=-3mm,yshifttext=-1mm}
}

\usepackage[ngerman]{babel}
\usepackage{inputenc}

\begin{document}

\begin{boxK}{Formelsammlung Methodenlehre}
Download der aktuellen Version unter \url{https://johannes-titz.github.io/formelsammlung/main.pdf}
\vspace{0.25cm}

Diese Formelsammlung wurde für Studierende der Psychologie entwickelt, ist aber vermutlich auch für Studierende anderer Sozialwissenschaften hilfreich. Die Formelsammlung basiert auf einer alten Sammlung der Professur für Forschungsmethoden und Evaluation in der Psychologie an der TU Chemnitz, die weder Autoren noch Lizenz enhielt. Für eine sinnvolle Nutzung und Weiterentwicklung ist eine Lizenz notwendig, weshalb wir das vorliegende Dokument unter CC BY-SA 4.0 veröffentlichen. Die Formelsammlung orientiert sich am Lehrbuch von Sedlmeier und Renkewitz (2018). Die Autoren der Formelsammlung sind: Johannes Titz, Nils Heimhuber, Feline Baumgärtel. Weitere Mitwirkende: Vivien Lungwitz, Annika Sternkopf. Für wertvolle Hinweise bedanken wir uns bei: Lea Haase.
\vspace{0.25cm}

Einige Hinweise vorab: Lateinische Buchstaben ($\bar{x}, s^2, s$) werden für die Stichprobe verwendet, griechische Buchstaben ($\mu, \sigma^2, \sigma$) für die Population. Ein Dach über einer Statistik ($\hat{\sigma}$) steht für eine Schätzung des Parameters. Abkürzungen:
\vspace{0.25cm}

\noindent\begin{tabular}{@{}ccc}
MWU: Mittelwertsunterschied & US: Unabhängige Stichproben & AS: Abhängige Stichproben\\
\end{tabular}
\end{boxK}

\begin{boxK}{Lagemaße}

\begin{multicols}{2}

\subsection*{Modalwert}

Der Modalwert ist der häufigste Wert aller Messwerte. Es kann mehrere Modal-Werte geben.


\subsection*{Mittelwert}
Auch bezeichnet als arithmetisches Mittel, für den Mittelwert der Population wird der Buchstabe $\mu$ verwendet.
\begin{equation*}
\bar{x}=\frac{1}{n}\sum\limits_{i=1}^n x_i
\end{equation*}

\subsection*{Median}

Der Median ist der Wert in der Mitte aller in einer Rangreihe geordneten Messwerte.

\begin{equation*}
\mathrm{Tiefe}_{\mathrm{Median}}=\frac{n+1}{2}
\end{equation*}

\subsection*{Quartile}
Die Werte müssen in eine Rangreihe gebracht werden. Für das untere Quartil wird die Tiefe von unten gezählt, fürs obere Quartil von oben. Quartile werden manchmal auch als die Quantile $Q_{25}, Q_{75}$ bezeichnet.
\begin{equation*}
\mathrm{Tiefe}_{\mathrm{Quartil}}=\frac{\lfloor\mathrm{Tiefe}_{\mathrm{Median}}\rfloor + 1}{2}
\end{equation*}

\subsection*{Zäune bei Boxplots}
Die Zäune werden benötigt um die Whiskers zu bestimmen. Die Whiskers sind durch tatsächlich vorkommende Werte repräsentiert. Von den Zäunen aus geht man in Richtung Box, bis man den ersten vorkommenden Wert findet.

\begin{align*}
  \mathrm{Zaun}_{\mathrm{unten}}&= Q_{25}-1,5 \cdot \mathrm{IQA}\\
  \mathrm{Zaun}_{\mathrm{oben}} &= Q_{75}+1,5 \cdot \mathrm{IQA}
\end{align*}


\end{multicols}
\end{boxK}


\begin{boxK}{Streuungsmaße}
\begin{multicols}{2}

\subsection*{Spannweite (Range)} 

\begin{equation*}
R=x_\mathrm{max}-x_\mathrm{min}
\end{equation*}

\subsection*{Varianz}

$$s^2=\frac{1}{n}\sum\limits_{i=1}^{n}(x_i-\bar{x})^2$$

Hinweis: In der mathematischen Statistik wird die \emph{Stichprobenvarianz} üblicherweise mit dem Nenner $n-1$ definiert,  da dieser Schätzer erwartungstreu für die Populationsvarianz ist. In den Sozialwissenschaften wird hingegen oft die Variante mit $n$ als \emph{Varianz der Stichprobe} bezeichnet. Hat man dagegen die gesamte Population, so wird auch die \emph{Populationsvarianz} mit $n$ im Nenner berechnet. 

\subsection*{Standardabweichung}
$$s=\sqrt{s^2}$$

\subsection*{Mittlerer absoluter Abstand}

$$\mathrm{MAD} = \frac{1}{n}\sum_{i=1}^n|x_i-\bar{x}|$$

\subsection*{Interquartilsabstand}

\begin{equation*}
\mathrm{IQA}=Q_{75}-Q_{25}
\end{equation*}
\vspace{0.1mm}

\subsection*{Schätzung für Populationsvarianz}

$$\hat{\sigma}^2=\frac{1}{n-1}\sum\limits_{i=1}^n(x_i-\bar{x})^2=\frac{n}{n-1}\cdot s^2$$

%\subsection*{Standardfehler}

%$$\hat{\sigma}=\sqrt{\hat{\sigma}^2}=\sqrt{\frac{n}{n-1}}\cdot s$$ 
\end{multicols}
\end{boxK}

\begin{boxK}{Transformation}
\begin{multicols}{2}
\subsection*{z-Standardisierung}
$$z_{i}= \frac{x_{i}-\bar{x}}{{s}}$$
$$\bar{z}=0$$
$$s_z=1$$

\subsection*{Zentrierung}

$c_{i} = x_i - \bar{x}$

\subsection*{IQ-Standardisierung}
$\mathrm{IQ_i} = z_i \cdot 15 + 100$\
\end{multicols}
\end{boxK}

\begin{boxK}{Standardfehler für Stichprobenverteilungen}
\begin{multicols}{2}
\subsection*{Mittelwerte}
%Mittelwerte
%$\hat{\sigma}^2_{\bar{x}}=\frac{\hat{\sigma}^2}{n}$ = \frac{s^2}{n-1}$

$$\hat{\sigma}_{\bar{x}}=\frac{\hat{\sigma}}{\sqrt{n}}=\frac{s}{\sqrt{n-1}}$$ 

\subsection*{MWU für US}
%Mittelwertsunterschiede für unabhängige Stichproben
wenn $n_A = n_B$: 
%
$$\hat{\sigma}_{\bar{x}_\mathrm{A}-\bar{x}_\mathrm{B}}=\sqrt{\hat{\sigma}^2_{\bar{x}_\mathrm{A}}+\hat{\sigma}^2_{\bar{x}_\mathrm{B}}}$$

wenn $n_A \neq n_B$
%
$$\hat{\sigma}_{\bar{x}_\mathrm{A}-\bar{x}_\mathrm{B}}=\sqrt{\frac{(n_A-1)\cdot\hat{\sigma}^2_\mathrm{A}+(n_\mathrm{B}-1)\cdot\hat{\sigma}^2_\mathrm{B}}{(n_\mathrm{A}-1)+(n_\mathrm{B}-1)}(\frac{1}{n_\mathrm{A}}+\frac{1}{n_\mathrm{B}})}$$

\subsection*{MWU für AS}
%
$$\hat{\sigma}_{\bar{\mathrm{diff}}}=\frac{\hat{\sigma}_{\mathrm{diff}}}{\sqrt{n}}$$ 

\subsection*{Binomialverteilung} 

$$\sigma_{\mathrm{Person}}=\sqrt{n p (1-p)}$$

$$\sigma_{\mathrm{Anteil}}=\sqrt{\frac{p (1-p)}{n}}$$ 

\end{multicols}
\end{boxK}

\begin{boxK}{Zusammenhangsmaße}

%\section{Zusammenhangsmaße}

\begin{multicols}{2}
\subsection*{Kovarianz}
$$\mathrm{cov}(x, y)=\frac{1}{n}\sum\limits_{i=1}^{n}(x_i-\bar{x})(y_i-\bar{y})$$

\subsection*{Korrelation}
Pearson-Korrelationskoeffizient \\
(Produkt-Moment-Korrelation)
$$r=\frac{\mathrm{cov}(x, y)}{s_x s_y}= \frac{1}{n}\sum\limits_{i=1}^{n}{z_{x_i}\cdot z_{y_i}}$$

\subsection*{Partialkorrelation}
$$r_{xy.z}=\frac{r_{xy}- r_{xz}\cdot r_{yz}}{\sqrt{1-r_{xz}^{2}}\cdot\sqrt{1-r_{yz}^{2}}}$$

\subsection*{Phi-Koeffizient}
$$\Phi=\frac{a\cdot d-b\cdot c}{\sqrt{(a+b)(c+d)(a+c)(b+d)}}$$

\begin{center}
\begin{tabular}{@{} r cc @{}}
\toprule
    & nein & ja \\
\midrule
nein & a & b \\
ja   & c & d \\
\bottomrule
\end{tabular}
\end{center}

\end{multicols}

\end{boxK}
\begin{boxK}{Einfache lineare Regression}
\begin{multicols}{2}
\subsection*{Regressionsgleichung}
\begin{align*}
\hat{y}_i &= a+b\cdot x_{i}\\
b &= \frac{\mathrm{cov}(x,y)}{s^2_x} = r \frac{s_y}{s_x}\\
a &= \bar{y}-b \bar{x}\\
y_i &= \hat{y}_i + e_i
\end{align*}
    
\subsection*{Determinationskoeffizient} 
$$R^2=\frac{s_{\hat{y}}^2}{s_y^2} $$

\subsection*{Standardschätzfehler}
$$s_e= \sqrt{\frac{\sum\limits_{i=1}^n(y_i-\hat{y}_i)^2}{n}} = s_y\cdot\sqrt{1-r^2}$$ 

$$\hat{\sigma}_e = \sqrt{\frac{\sum\limits_{i=1}^n(y_i-\hat{y}_i)^2}{n-2}}$$

\end{multicols}
\end{boxK}

\begin{boxK}{Multiple Regression}

 \begin{multicols}{2}
In den folgenden Formeln wird zur Veranschaulichung der Fall mit zwei Prädiktoren dargestellt. 
Für mehr als zwei Prädiktoren lassen sich die Regressionsgleichung, die Beziehung zwischen $b$ und $\beta$, 
die Berechnung von $b_0$ sowie die Formeln für den Standardschätzfehler analog erweitern. 
Die Berechnung der $\beta$-Gewichte und von $R^2$ erfordert jedoch eine Matrixschreibweise, die nicht überall unterichtet wird.

    \subsection*{Regressionsgleichung}
für Originalwerte: $$\hat{y}_i=b_0+b_1 \cdot x_{1i} + b_2 \cdot x_{2i}$$
wobei: 
\begin{align*}
b_1&=\beta_1 \cdot \frac{s_y}{s_{x_{1}}}\\
b_2&=\beta_2 \cdot \frac{s_y}{s_{x_{2}}}\\
b_0&=\bar{y}-b_1 \cdot \bar{x}_{1}-b_2 \cdot \bar{x}_2\\
\end{align*}

für z-standardisierte Werte: $$\hat{z}_{y_i}=\beta_1\cdot z_{x_{1i}} + \beta_2\cdot z_{x_{2i}}$$ \\
wobei: $$\beta_1=\frac{r_{yx_{1}}-r_{yx_{2}}\cdot r_{x_{1}x_{2}}}{1-r^2_{x_{1}x_{2}}}$$ \ $$\beta_2=\frac{r_{yx_{2}}-r_{yx_{1}} \cdot r_{x_{1}x_{2}}}{1-r^2_{x_{1}x_{2}}}$$\\

\columnbreak
    \subsection*{Gütemaße}

Multipler Determinationskoeffizient
$$R^2=\beta_1 \cdot r_{yx_{1}} + \beta_2 \cdot r_{yx_{2}}$$ 
Multipler Korrelationskoeffizient
$$R=\sqrt{R^2}$$
Standardschätzfehler für Originalwerte
$$s_e=\sqrt{\frac{\sum\limits_{i=1}^n(y_i-\hat{y}_i)^2}{n}} = s_y \sqrt{1-R^2}$$

$$\hat{\sigma}_e = \sqrt{\frac{\sum\limits_{i=1}^n(y_i-\hat{y}_i)^2}{n-k-1}}$$

$k$ ist die Anzahl der Prädiktoren
\vspace{1em}

Standardschätzfehler für $z$-standardisierte Werte
$$s_e=\sqrt{1-R^2}$$

\end{multicols}
\end{boxK}


\begin{boxK}{Konfidenzintervalle}
%\section{Konfidenzintervalle}

\begin{multicols}{2}

\subsection*{Binomialverteilung (Approximation)}
Die folgende Approximation über die Normalverteilung ist nur nutzbar, wenn $\sigma^2_{\mathrm{Person}} = n \cdot p \cdot (1-p)>9$ \\\\

für Anteile
$$\hat{p} \pm \sigma_{\mathrm{Anteil}}\cdot z_{\mathrm{Konfidenz}}$$

für Personen/Objekte
$$\hat{p} \cdot n \pm \sigma_{\mathrm{Person}}\cdot z_{\mathrm{Konfidenz}}$$

\subsection*{Mittelwert} 
$$\bar{x} \pm \hat{\sigma}_{\bar{x}} \cdot t_{\mathrm{df, Konfidenz}}$$

\subsection*{MWU, US} 
$$(\bar{x}_\mathrm{A}-\bar{x}_\mathrm{B}) \pm \hat{\sigma}_{\bar{x}_\mathrm{A}-\bar{x}_\mathrm{B}}\cdot t_{{\mathrm{df}}\mathrm{, Konfidenz}}$$

\subsection*{MWU, AS}
$${\overline{\mathrm{diff}}} \pm \hat{\sigma}_{\overline{\mathrm{diff}}}\cdot t_{{\mathrm{df}}
\mathrm{, Konfidenz}}$$

\end{multicols}
\end{boxK}


\begin{boxK}{t-Test}

%\section{t-Test}
\begin{multicols}{3}

\subsection*{Mittelwert gegen Konstante} 
wobei c (constant) vorgegeben
$$t=\frac{\bar{x}-c}{\hat{\sigma}_{\bar{x}}}$$
$$\mathrm{df}=n-1$$

\begin{center}
\subsection*{MWU, US} 

\begin{align*}
t&=\frac{\bar{x}_A-\bar{x}_B}{\hat{\sigma}_{\bar{x}_A-\bar{x}_B}}\\
\mathrm{df}&=(n_A-1)+(n_B-1)
\end{align*}
\end{center}
Hinweis: Berechnung der $\mathrm{df}$ gilt nur bei Gleichheit der Populationsvarianzen. 

\begin{center}
\subsection*{MWU, AS}
$$t_{\mathrm{AS}}=\frac{\bar{d}}{\hat{\sigma}_{\bar{d}}}$$
$$\mathrm{df}_{\mathrm{AS}}=n-1$$
\end{center}

\end{multicols}
\end{boxK}

\begin{boxK}{ANOVA US}
\begin{multicols}{2}
\subsection*{F-Wert}
$$F(\mathrm{df}_{\mathrm{zw}},\mathrm{df}_{\mathrm{inn}})=\frac{\hat{\sigma}^2_{\mathrm{zw}}}{\hat{\sigma}^2_ {\mathrm{inn}}}$$\

\subsection*{Populationsvarianzen}
$$\hat{\sigma}^2_{\mathrm{zw}}=\frac{\mathrm{QS}_{\mathrm{zw}}}{\mathrm{df}_{\mathrm{zw}}}$$ $$\hat{\sigma}^2_{\mathrm{inn}}=\frac{\mathrm{QS}_{\mathrm{inn}}}{\mathrm{df}_{\mathrm{inn}}}$$\\

\subsection*{Quadratsummenzerlegung}
$$\mathrm{QS}_{\mathrm{ges}}=\mathrm{QS}_{\mathrm{zw}}+\mathrm{QS}_{\mathrm{inn}}$$\

\subsection*{Quadratsummen}
$$\mathrm{QS}_{\mathrm{zw}}=\sum\limits_{j}^k n_j \cdot(\bar{x}_j-\bar{\bar{x}})^2$$\
$$\mathrm{QS}_{\mathrm{inn}}=\sum\limits_{j}^k \sum\limits_{i}^{n_j}(x_{ij}-\bar{x}_j)^2$$\

\subsection*{Freiheitsgrade}
\begin{align*}
\mathrm{df}_{\mathrm{zw}}&=k-1\\
\mathrm{df}_{\mathrm{inn}}&=\sum\limits_{j}^k(n_j-1) = N-k
\end{align*}

\subsection*{Notation}

\begin{tabular}{rl}
$k$: &Anzahl der Bedingungen \\
$n_j$: &Anzahl der Messwerte pro Bedingung $j$ \\
$N$: &Gesamtzahl der Messwerte\\
\end{tabular}
\end{multicols}
\end{boxK}

\begin{boxK}{ANOVA AS}
\begin{multicols}{2}
\subsection*{F-Wert}
$$F=\frac{\hat{\sigma}_{\mathrm{UV}}^2}{\hat{\sigma}_{\mathrm{res}}^2}$$

\subsection*{Populationsvarianzen}
$$\hat{\sigma}_{\mathrm{UV}}^2= \frac{\mathrm{QS}_{\mathrm{UV}}}{\mathrm{df}_{\mathrm{UV}}}$$
$$\hat{\sigma}_{\mathrm{res}}^2= \frac{\mathrm{QS}_{\mathrm{res}}}{\mathrm{df}_{\mathrm{res}}}$$

\subsection*{Quadratsummen}
\begin{align*}
\mathrm{QS}_{\mathrm{ges}} &= \sum_{j}^k \sum_{i}^n (x_{ji}-\bar{\bar{x}})^2 \\
\mathrm{QS}_{\mathrm{UV}} &= \sum_{j}^k n\cdot (\bar{x}_{j}-\bar{\bar{x}})^2 \\
\mathrm{QS}_{\mathrm{Person}} &= \sum_{i}^n k\cdot (\bar{x}_{i}-\bar{\bar{x}})^2 \\
\mathrm{QS}_{\mathrm{res}} &= \mathrm{QS}_{\mathrm{ges}} - \mathrm{QS}_{\mathrm{UV}} - \mathrm{QS}_{\mathrm{Person}}
\end{align*}

\subsection*{Quadratsummenzerlegung}
$$\mathrm{QS}_{\mathrm{ges}}=\mathrm{QS}_{\mathrm{UV}}+ \mathrm{QS}_{\mathrm{Person}}+\mathrm{QS}_{\mathrm{res}}$$

\subsection*{Freiheitsgrade} 
\begin{align*} 
\mathrm{df}_{\mathrm{gesamt}} &= N-1 \\
\mathrm{df}_{\mathrm{UV}} &= k-1 \\
\mathrm{df}_{\mathrm{Person}} &= n-1 \\
\mathrm{df}_{\mathrm{res}} &= (k-1)(n-1) \
\end{align*}

\subsection*{Notation}
\begin{tabular}{rl}
$k$: & Anzahl der Bedingungen \\
$n$: & Anzahl Messwerte pro Bedingung \\
$N$: & Gesamtzahl der Messwerte \\
$\bar{\bar{x}}$: & Gesamtmittelwert
\end{tabular}
\end{multicols}
\end{boxK}

% gecheckt mit chat-gpt
\begin{boxK}{Zweifaktorielle Varianzanalyse}
\begin{center}
 
Hinweis: gilt nur bei balancierten Designs (gleiche Stichprobengröße)
\end{center}

\begin{multicols}{2}
\subsection*{F-Werte} 
$$F_A=\frac{\hat{\sigma}_A^2}{\hat{\sigma}^2_{\mathrm{inn}}}$$ $$F_B=\frac{\hat{\sigma}_B^2}{\hat{\sigma}^2_{\mathrm{inn}}}$$ $$F_{A\times B}=\frac{\hat{\sigma}_{A\times B}^2}{\hat{\sigma}^2_{\mathrm{inn}}}$$

\subsection*{Populationsvarianzen}
$$\hat{\sigma}_A^2=\frac{\mathrm{QS}_{A}}{\mathrm{df}_{A}}$$ $$\hat{\sigma}_B^2=\frac{\mathrm{QS}_{B}}{\mathrm{df}_{B}}$$  $$\hat{\sigma}_{A\times B}^2=\frac{\mathrm{QS}_{A\times B}}{\mathrm{df}_{A\times B}}$$
$$\hat{\sigma}^2_ {\mathrm{inn}}=\frac{\mathrm{QS}_{\mathrm{inn}}}{\mathrm{df}_{\mathrm{inn}}}$$

\subsection*{Freiheitsgrade}
\begin{align*}
%\mathrm{df}_{\mathrm{ges}}&=N-1\\
%\mathrm{df}_{\mathrm{inn}}&=k\cdot m (n-1)\\
\mathrm{df}_{\mathrm{inn}}&=\sum\limits_j^k \sum\limits_l^m (n_{jl}-1) = N-km \\
\mathrm{df}_{\mathrm{A}}&=k-1\\
\mathrm{df}_{\mathrm{B}}&=m-1\\
\mathrm{df}_{\mathrm{AxB}}&=(k-1)(m-1)\\
\end{align*}

\subsection*{Quadratsummen}
\begin{align*}
\mathrm{QS}_{\mathrm{ges}}&=\mathrm{QS}_A+\mathrm{QS}_B+ \mathrm{QS}_{A\times B}+\mathrm{QS}_\mathrm{inn}\\
\mathrm{QS}_{\mathrm{ges}}&=\sum\limits_j^k \sum\limits_l^m \sum\limits_i^n (x_{ijl}-\bar{\bar{x}})^2\\
\mathrm{QS}_{\mathrm{inn}}&=\sum\limits_j^k \sum\limits_l^m \sum\limits_i^n(x_{ijl}-\bar{x}_{jl})^2\\
\mathrm{QS}_A&=n \cdot m \sum\limits_j^k (\bar{x}_{j\cdot}-\bar{\bar{x}})^2\\
\mathrm{QS}_B&= n \cdot k \sum\limits_l^m (\bar{x}_{\cdot l}-\bar{\bar{x}})^2\\
\mathrm{QS}_{A\times B}&= n \sum\limits_{j=1}^k \sum\limits_{l=1}^m \left( \bar{x}_{jl} - \bar{x}_{j\cdot} - \bar{x}_{\cdot l} + \bar{\bar{x}} \right)^2 \\
\mathrm{QS}_{A\times B}&=\mathrm{QS}_{\mathrm{ges}}-\mathrm{QS}_{\mathrm{A}}-\mathrm{QS}_{\mathrm{B}}-\mathrm{QS}_{\mathrm{inn}}\\
\end{align*}

\subsection*{Notation}
\begin{tabular}{rl}
$N$: &Gesamtzahl der Messwerte der Untersuchung \\
$n$: &Anzahl der Messwerte pro Bedingung \\
$k$: &Anzahl der Stufen des Faktors $A$\\
$m$: &Anzahl der Stufen des Faktors $B$\\
$\bar{\bar{x}}$: &Gesamtmittelwert\\
$\bar{x}_{j\cdot}$: &Mittelwert für Faktor $A$, Bedingung $j$\\
$\bar{x}_{\cdot l}$: &Mittelwert für Faktor $B$, Bedingung $l$\\
\end{tabular}
\end{multicols}


\end{boxK}


\begin{boxK}{Kontrastanalyse US}
\begin{multicols}{2}

$$F_{\mathrm{Kontrast}}=\frac{\hat{\sigma}^2_{\mathrm{Kontrast}}}{\hat{\sigma}^2_{\mathrm{inn}}}$$ 
\begin{center}
    oder
\end{center}
$$F=t^2$$
$$|t_{\mathrm{Kontrast}}|=\sqrt{F_{\mathrm{Kontrast}}}$$
\subsection*{Populationsvarianz}

\begin{align*}
\hat{\sigma}^2_{\mathrm{Kontrast}}&=\frac{\mathrm{QS}_{\mathrm{Kontrast}}}{\mathrm{df}_{\mathrm{Kontrast}}}\\
\hat{\sigma}^2_{\mathrm{inn}}&=\frac{\mathrm{QS}_\mathrm{inn}}{\mathrm{df}_\mathrm{inn}}\\
\mathrm{oder}\hspace{4mm}\hat{\sigma}^2_{\mathrm{inn}}&=\frac{\sum\limits_{j=1}^k \hat{\sigma}_j^2}{k}\\
\end{align*}

\subsection*{Quadratsummen}
$$\mathrm{QS}_{\mathrm{Kontrast}}=\frac{\left(\sum\limits_{i=1}^k \lambda_i\bar{x}_i\right)^2}{\sum\limits_{i=1}^k \frac{\lambda_i^2}{n_i}}$$
\subsection*{Freiheitsgrade}
$$\mathrm{df}_{\mathrm{Kontrast}}=1$$
$$\mathrm{df}_{\mathrm{inn}}=N-k$$

\subsection*{Notation}
\begin{tabular}{rl}
$k$: &Gruppe \\
$n_i$: &Anzahl der Messwerte pro Bedingung i\\
$n_j$: &Anzahl der Messwerte pro Bedingung j
\end{tabular}
\end{multicols}
\end{boxK}

\begin{boxK}{Kontrastanalyse für AS}
\begin{multicols}{2}
(ohne Subgruppen)
\begin{align*}
t_{\mathrm{Kontrast}} &= \frac{\bar{L}}{\sqrt{\frac{\hat{\sigma}^2_{L}}{n}}} \\
L_i &= \sum_{j=1}^k (x_{ij} \cdot \lambda_{j}) \\
\hat{\sigma}^2_{L} &= \frac{1}{n-1} \sum_{i=1}^n (L_i - \bar{L})^2 \\
\mathrm{df}_{\mathrm{Kontrast}} &= n-1
\end{align*}

\subsection*{Notation}
\begin{tabular}{rl}
$\bar{L}$: &Mittelwert $L-$Werte
\\$\hat{\sigma}^2_i$: &geschätzte Populationsvarianz der Gruppe i
\\$n$: &Anzahl der Objekte/Personen, \\
&für die mehrere Messungen durchgeführt
\end{tabular}
\end{multicols}
\end{boxK}


\begin{boxK}{Nonparametrische Verfahren}
%\section{Nonparametrische Verfahren}
\begin{multicols}{2}

\subsection*{$\chi^2$-Anpassungstests}
für eine Variable
\begin{align*}
\chi^2&=\sum\limits_i^k \frac{(f_{b,i}-f_{e,i})^2}{f_{e,i}}\\
f_{e,i}&=N \cdot P_i\\
\mathrm{df}&=k-1\\
\end{align*}
für zwei Variablen
\begin{align*}
\chi^2&=\sum\limits_{i}^k \sum\limits_{j}^m \frac{(f_{b,ij}-f_{e,ij})^2}{f_{e,ij}}\\
f_{e,ij}&=N \cdot P_{ij}\\
\mathrm{df}&=k \cdot m-1\\
\end{align*}

\subsection*{$\chi^2$-Unabhängigkeitstest}
\begin{align*}
\chi^2&=\sum\limits_{i}^k \sum\limits_{j}^m \frac{(f_{b,ij}-f_{e,ij})^2}{f_{e,ij}}\\
f_{e,ij}&=\frac{Z_i \cdot S_j}{N}\\
\mathrm{df}&=(k-1)\cdot(m-1)\\
\end{align*}
Bei 2 dichotomen Variablen: $\chi^2=\phi^2 \cdot N$\\

\columnbreak
\subsection*{U-Test nach Mann und Whitney (Wilcoxon Rangsummen-Test)}
\begin{enumerate}
    \item Messwerte insgesamt in Rangreihe bringen und jedem Messwert einen Rangplatz zuweisen
    \item T$_1$: Summe der Rangplätze der Gruppe 1
            T$_2$: Summe der Rangplätze der Gruppe 2
    \item $U=n_1\cdot n_2+\frac{n_1\cdot(n_1+1)}{2}-T_1$ \\
            $U^{\prime}=n_1 \cdot n_2+\frac{n_2 \cdot(n_2+1)}{2}-T2 = n_1 \cdot n_2-U$
    \item Prüfgröße für die bei uns verwendete Tabelle ist der kleinere der beiden U-Werte
\end{enumerate}



\subsection*{Wilcoxon-Test für AS (Vorzeichenrangtest)}
\begin{enumerate}
    \item Differenzen der Messwertpaare bilden
    \item den Beträgen der Differenzen Rangplatz zuweisen (beim kleinsten Betrag mit Rangplatz 1 beginnen)
    \item $T_-$: Summe der Rangplätze der negativen Differenzen
    $T_+$: Summe der Rangplätze der positiven Differenzen
    \item Prüfgröße für die Tabelle ist der kleinere der beiden T-Werte
\end{enumerate}

\subsection*{Notation}
\begin{tabular}{rl}
    $f_{b,i}$:&beobachtete Häufigkeit\\
    $f_{e,i}$:&erwartete Häufigkeit \\
    $P_i$: &in Merkmalsausprägung erwarteter Anteil\\
    $N$: &Anzahl der Untersuchungsteilnehmer\\
    $k$ ; $m$: &Anzahl der Merkmalsausprägungen beider\\ 
    &Merkmale\\
    $Z_i$: &Zeilenhäufigkeit \\
    $S_j$: &Spaltenhäufigkeit \\
    $\phi$: &Phi-Koeffizient\\
 \end{tabular}
 
\end{multicols}
\end{boxK}




\begin{boxK}{Faktorenanalyse}
\begin{multicols}{2}
\begin{enumerate}
    \item Faktorladung: $a_{mk}=r(m,k)$
    \item Eigenwert: $\lambda_k=\sum\limits_{m=1}^M a_{mk}^2$
    \item Kommunalität: $h_m^2=\sum\limits_{k=1}^f a_{mk}^2$
    \item durch Faktor aufgeklärter Varianzanteil: $\frac{\lambda_k}{M}$
\end{enumerate}

\subsection*{Notation}
\begin{tabular}{rl}
$m$: &Variable, \\
$k$: &Anzahl der Faktoren, \\
$i$: &Person/Objekt, \\ 
$M$: &Anzahl der Variablen = Gesamtvarianz\\
$r$: &Korrelation\\
$f$: &Anzahl der ausgewählten Faktoren
\end{tabular}
\end{multicols}
\end{boxK}


\begin{boxK}{Clusteranalyse}
\begin{multicols}{2}
\subsection*{Nominalskalierte Variablen}

\vspace{0.5cm}
\begin{center}
\begin{tabular}{lrrr}
   \toprule & & \multicolumn{2}{c}{Fall x} \\
    \midrule
    & & $+$ & $-$ \\
    Fall y & $+$ & a & c \\
    & $-$ & b & d \\
   \bottomrule
\end{tabular}
\end{center}

\vspace{0.5cm}
Tanimoto-Koeffizient
$$T=\frac{a}{a+b+c}$$
M-Koeffizient
$$M=\frac{a+d}{a+b+c+d}$$

\columnbreak
\subsection*{Intervallskalierte Variablen}

\begin{tabular}{rl}
$a,b$:& Fälle\\
$J$:& Anzahl der Variablen (Dimensionen)\\
$r$:& Minkowski-Konstante
\end{tabular}
\vspace{1em}

Minkowski-Metrik
$$d_{a,b}=\left[\sum\limits_{j=1}^J |x_{aj}-x_{bj}|^r \right]^{\frac{1}{r}}$$

Euklidische Distanz
$$d_{a,b}=\sqrt{\sum\limits_{j=1}^J|x_{aj}-x_{bj}|^2}$$

Manhattan-Distanz / City-Block-Metrik
$$d_{a,b}=\sum\limits_{j=1}^J|x_{aj}-x_{bj}|$$

\end{multicols}
\end{boxK}

\input{13standard.tex}
\begin{boxK}{Effektgrößen aus Signifikanztests}
\begin{multicols}{2}
\subsection*{Konventionen nach Cohen}
"These values are necessarily somewhat arbitrary, but were chosen so as to seem reasonable. The reader can render his own judgment as to their reasonableness."
\vspace{0.5cm} (Cohen, 1962)

\begin{tabular}{lccc}
    \toprule
    & \multicolumn{1}{r}{d/g} & \multicolumn{1}{r}{r/$w$/$\phi$} & \multicolumn{1}{r}{$\eta^2$} \\
    \midrule
    klein & $\pm 0.2$ & $\pm 0.1$ & $0.01$ \\
    mittel & $\pm 0.5$ & $\pm 0.3$ & $0.06$ \\
    groß & $\pm 0.8$ & $\pm 0.5$ & $0.14$ \\
    \bottomrule
\end{tabular}
\vspace{1cm}

\section*{$t$-Tests}
Einstichprobenfall
$$g=\frac{t}{\sqrt{n}}$$\\
$$d=\frac{t}{\sqrt{\mathrm{df}}}$$\\
%wenn $n_\mathrm{A}=n_\mathrm{B}$: 
%$$d=\frac{2\cdot t_\mathrm{US}}{\sqrt{\mathrm{df}}}$$

Zwei unabhängige Stichproben
$$g=t_\mathrm{US}\cdot\sqrt{\frac{n_\mathrm{A}+n_\mathrm{B}}{n_\mathrm{A}\cdot n_\mathrm{B}}}$$
%wenn $n_\mathrm{A}=n_\mathrm{B}$: 
%$$g=\frac{2\cdot t_\mathrm{US}}{\sqrt{2n}}$$
$$d=t_\mathrm{US}\cdot\frac{n_\mathrm{A}+n_\mathrm{B}}{\sqrt{\mathrm{df}}\cdot\sqrt{n_\mathrm{A}\cdot n_\mathrm{B}}}$$
$$r=\sqrt{\frac{t^2_\mathrm{US}}{t^2_\mathrm{US}+\mathrm{df}}}$$
Letzteres gilt auch für $F$, wenn $\mathrm{df}_{\mathrm{zw}}$ 1 sind, da in diesem Fall $F=t^2$.\\


Zwei abhängige Stichproben
$$g=\frac{t_\mathrm{AS}}{\sqrt{n}}$$
$$d=\frac{t_\mathrm{AS}}{\sqrt{\mathrm{df}}}$$


\subsection*{Kontrastanalyse US}
aus F-Test
$$r_{\mathrm{effect size}}=\sqrt{\frac{F_{\mathrm{Kontrast}}}{F_{\mathrm{zw}} \cdot {\mathrm{df}}_{\mathrm{zw}}+{\mathrm{df}}_{\mathrm{inn}}}}$$

$$r_{\mathrm{contrast}} = \sqrt{\frac{F_{\mathrm{Kontrast}}}{F_{\mathrm{Kontrast}}+\mathrm{df_{\mathrm{inn}}}}} = \sqrt{\frac{t^2_{\mathrm{Kontrast}}}{t^2_{\mathrm{Kontrast}}+\mathrm{df}}} $$
\subsection*{Kontrastanalyse AS}
aus t-Test
$$g=\frac{t}{\sqrt{n}}$$

\subsection*{$\chi^2$-Tests}
$$w=\sqrt{\sum\limits_{i=1}^k\frac{(P_{b,i}-P_{e,\;i})^2}{P_{e,i}}}$$
aus Ergebnis $\chi^2$-Test:
$$w=\sqrt{\frac{\chi^2}{N}}$$
aus 2 dichotomen Variablen
$$w=\sqrt{\frac{\chi^2}{N}}=\phi$$


\subsection*{ANOVA}

US
\begin{align*}
\eta^2&=\frac{F \cdot \mathrm{df}_{\mathrm{zw}}}{F \cdot \mathrm{df}_{\mathrm{zw}}+\mathrm{df}_{inn}}\\
\eta^2&=\frac{\mathrm{QS}_{\mathrm{zw}}}{\mathrm{QS}_{\mathrm{ges}}}\\
\end{align*}
AS
\begin{align*}
\eta^2&=\frac{\mathrm{QS}_{\mathrm{UV}}}{\mathrm{QS}_{\mathrm{ges}}}\\
\eta^2_p&=\frac{\mathrm{QS}_{\mathrm{UV}}}{\mathrm{QS}_{\mathrm{UV}}+\mathrm{QS}_{\mathrm{res}}}\\
\end{align*}
mehrfaktoriell
\begin{align*}
\eta^2=\frac{\mathrm{QS}_{\mathrm{Effekt}}}{\mathrm{QS}_{\mathrm{ges}}}\\
\eta^2_p=\frac{\mathrm{QS}_{\mathrm{Effekt}}}{\mathrm{QS}_{\mathrm{Effekt}}+\mathrm{QS}_{\mathrm{inn}}}\\
\end{align*}

\end{multicols}
\end{boxK}

\begin{boxK}{Effektgrößen aus Rohwerten}
\begin{multicols}{3}

\begin{align*}
d &= \frac{\bar{x}-c}{\sigma_x} \\
d &= \frac{\bar{x_1}-\bar{x_2}}{\sqrt{\frac{(\sigma_{1}^2+\sigma_{2}^2)}{2}}}\\
g &= \frac{\bar{x}-c}{s_{x}} \\
g &= \frac{\bar{x}_1-\bar{x}_2}{s}\\
\end{align*}


\begin{math}
\begin{aligned}
s &= \sqrt{\frac{(n_1-1)\cdot s_1^2 + (n_2-1) \cdot s_2^2}{n_1+n_2-2}}
\end{aligned}
\end{math}

$$r_{\mathrm{effect size}}=\frac{\frac{1}{n}\sum\limits_i^n(x_i-\bar{x})\cdot(\lambda_i-\bar{\lambda})}{s_x \cdot s_{\lambda}}$$
$$r_{\mathrm{contrast}}= \frac{\mathrm{QS_{Kontrast}}}{\mathrm{QS_{Kontrast}}+\mathrm{QS_{Nicht-Kontrast}}}$$

$$r^2_{\mathrm{alerting}} = \frac{\mathrm{QS_{Kontrast}}}{\mathrm{QS_{zw}}}$$

$$g=\frac{\bar{L}}{\hat{\sigma}_{L}}$$

\end{multicols}
\end{boxK}

\begin{boxK}{Odds Ratio (OR)}
\begin{multicols}{2}

$$\mathrm{OR}= \frac{\frac{a}{c}}{\frac{b}{d}} = \frac{a \cdot d}{b \cdot c}$$\\
\begin{tabular}{@{} r ll @{}}
\toprule
    & Risikofaktor & ohne Risikofaktor \\
\midrule
Krankheit & a & b \\
Keine Krankheit   & c & d \\
\bottomrule
\end{tabular}
\end{multicols}
\end{boxK}

\begin{boxK}{Effektgrößen aus Effektgrößen}
\begin{multicols}{2}

\subsection*{Abstandsmaße aus Abstandsmaßen}
$$d=g\cdot\sqrt{\frac{n}{\mathrm{df}}}$$
$$g=d\cdot\sqrt{\frac{df}{n}}$$

\subsection*{Korrelationen aus Abstandsmaßen}
\begin{align*}
r&=\frac{d}{\sqrt{d^2+\frac{1}{p\cdot q}}}\\
r&=\sqrt{\frac{g^2(n_A\cdot n_B)}{g^2(n_A\cdot n_B)+(n_A+n_B)\mathrm{df}}}\\
\end{align*}
\subsection*{Abstandsmaße aus Korrelationen}
\begin{align*}
d&=\frac{r}{\sqrt{1-r^2}}\cdot\sqrt{\frac{1}{p\cdot q}}\\
g&=\frac{r}{\sqrt{1-r^2}}\cdot\sqrt{\frac{(n_A+n_B)\cdot df}{n_A\cdot n_B}}\\
\end{align*}

wobei $$p=\frac{n_A}{n_A+n_B}$$ und $$q=\frac{n_B}{n_A+n_B}$$

\end{multicols}
\end{boxK}


\begin{boxK}{Meta-Analyse}
  Mittlere, an der Stichprobe gewichtete, Effektgröße: $\bar{r} = \dfrac{\sum_{i=1}^n(N_i r_i)}{\sum_{i=1}^n N_i}$
    %\subsection{}
\end{boxK}

\begin{boxK}{Wahrscheinlichkeitstheorie}

\begin{multicols}{2}

\subsection*{Bedingte Wahrscheinlichkeit}

Die bedingte Wahrscheinlichkeit von $A$ unter der Bedingung $B$ wird notiert als $P(A | B)$ und berechnet sich durch:

\[
P(A | B) = \frac{P(A \cap B)}{P(B)}
\]

Hierbei ist $P(A \cap B)$ die Wahrscheinlichkeit, dass sowohl $A$ als auch $B$ eintreten.

\subsection*{Additionssatz der Wahrscheinlichkeit}

Die Wahrscheinlichkeit, dass mindestens eines der Ereignisse $A$ oder $B$ eintritt, ergibt sich zu:

\[
P(A \cup B) = P(A) + P(B) - P(A \cap B)
\]

Sind $A$ und $B$ \emph{disjunkt} (d.\,h. $A \cap B = \emptyset$), so gilt die \emph{Summenregel}:
\[
P(A \cup B) = P(A) + P(B)
\]

\subsection*{Allgemeines Inklusions-Exklusions-Prinzip}

Für drei Ereignisse $A$, $B$ und $C$ gilt:

\[
\begin{aligned}
P(A \cup B \cup C)
&= P(A) + P(B) + P(C) \\
&\quad - [P(A \cap B) + P(A \cap C) + P(B \cap C)] \\
&\quad + P(A \cap B \cap C)
\end{aligned}
\]

Das Prinzip verallgemeinert sich für $n$ Ereignisse abwechselnd mit Addition und Subtraktion der Schnittwahrscheinlichkeiten aller $k$-fachen Überlappungen.

\subsection*{Produktregel}

Aus der Definition der bedingten Wahrscheinlichkeit folgt:

\[
P(A \cap B) = P(A | B) \cdot P(B)
\]

Alternativ kann die Produktregel auch in umgekehrter Reihenfolge geschrieben werden:

\[
P(A \cap B) = P(B | A) \cdot P(A)
\]

\subsection*{Stochastische Unabhängigkeit}

Zwei Ereignisse $A$ und $B$ heißen \emph{stochastisch unabhängig}, wenn das Eintreten des einen keinen Einfluss auf die Wahrscheinlichkeit des anderen hat:

\[
P(A | B) = P(A)
\quad \text{bzw.} \quad
P(B | A) = P(B)
\]

Dann folgt unmittelbar:

\[
P(A \cap B) = P(A) \cdot P(B)
\]

\subsection*{Satz der totalen Wahrscheinlichkeit}

Sei $\{B_1, B_2, \dots, B_n\}$ eine Zerlegung des Ergebnisraums in paarweise disjunkte Ereignisse mit $P(B_i) > 0$. Dann gilt:

\[
P(A) = \sum_{i=1}^n P(A | B_i) \cdot P(B_i)
\]

\subsection*{Wahrscheinlichkeitsrevision (Bayes)}

\[
P(A | B) = \frac{P(A) \cdot P(B | A)}{P(A) \cdot P(B | A) + P(A^{\mathrm{c}}) \cdot P(B | A^{\mathrm{c}})}
\]

\end{multicols}
\end{boxK}


%
\begin{boxK}{t-Verteilung}
In der Tabelle finden sich die kritischen $t$‐Werte. Das Signifikanzniveau wird durch die Fläche angegeben. Beim einseitigen Testen auf dem 5\% - Niveau beträgt die relevante Fläche 0,95; beim zweiseitigen Testen entsprechend 0,975. Der empirische $t$‐Wert muss gleich groß oder größer sein als der kritische $t$-Wert aus der Tabelle, um auf dem entsprechenden Niveau signifikant zu sein.



\vspace{1 cm}
\begin{center}
\begin{tabular}{|c|c|c|c|c|c|c|c|}
    \hline
    df & 0,8 & 0,85 & 0,9 & 0,95 & 0,975 & 0,99 & 0,995 \\
    \hline
    1 & 1,377 & 1,964 & 3,078 & 6,314 & 12,706 & 31,821 & 63,657 \\
    2 & 1,001 & 1,386 & 1,886 & 2,92 & 4,303 & 6,965 & 9,925 \\
    3 & 0,978 & 1,25 & 1,638 & 2,353 & 3,182 & 4,541 & 5,841 \\
    4 & 0,941 & 1,19 & 1,533 & 2,132 & 2,776 & 3,747 & 4,604 \\
    5 & 0,92 & 1,156 & 1,476 & 2,015 & 2,571 & 3,365 & 4,032 \\
    6 & 0,906 & 1,134 & 1,44 & 1,943 & 2,447 & 3,143 & 3,707 \\
    7 & 0,896 & 1,119 & 1,415 & 1,895 & 2,305 & 2,998 & 3,5 \\
    8 & 0,889 & 1,108 & 1,397 & 1,86 & 2,306 & 2,896 & 3,355 \\
    9 & 0,883 & 1,1 & 1,383 & 1,833 & 2,262 & 2,821 & 3,25 \\
    10 & 0,879 & 1,093 & 1,372 & 1,813 & 2,228 & 2,764 & 3,169 \\
    11 & 0,876 & 1,088 & 1,363 & 1,796 & 2,201 & 2,718 & 3,106 \\
    12 & 0,873 & 1,083 & 1,356 & 1,782 & 2,179 & 2,681 & 3,055 \\
    13 & 0,87 & 1,079 & 1,35 & 1,771 & 2,16 & 2,651 & 3,012 \\
    14 & 0,868 & 1,076 & 1,345 & 1,761 & 2,145 & 2,625 & 2,977 \\
    15 & 0,866 & 1,074 & 1,341 & 1,753 & 2,131 & 2,602 & 2,947 \\
    16 & 0,865 & 1,071 & 1,337 & 1,746 & 2,12 & 2,584 & 2,921 \\
    17 & 0,863 & 1,069 & 1,333 & 1,74 & 2,11 & 2,567 & 2,898 \\
    18 & 0,862 & 1,067 & 1,33 & 1,734 & 2,101 & 2,552 & 2,878 \\
    19 & 0,861 & 1,066 & 1,328 & 1,729 & 2,093 & 2,54 & 2,861 \\
    20 & 0,86 & 1,064 & 1,325 & 1,725 & 2,086 & 2,528 & 2,845 \\
    21 & 0,859 & 1,063 & 1,323 & 1,721 & 2,08 & 2,518 & 2,831 \\
    22 & 0,858 & 1,061 & 1,321 & 1,717 & 2,074 & 2,508 & 2,819 \\
    23 & 0,858 & 1,06 & 1,319 & 1,714 & 2,069 & 2,5 & 2,807 \\
    24 & 0,857 & 1,059 & 1,318 & 1,711 & 2,064 & 2,492 & 2,797 \\
    25 & 0,856 & 1,058 & 1,316 & 1,708 & 2,06 & 2,485 & 2,787 \\
    26 & 0,856 & 1,058 & 1,315 & 1,706 & 2,056 & 2,479 & 2,779 \\
    27 & 0,855 & 1,057 & 1,314 & 1,703 & 2,052 & 2,473 & 2,771 \\
    28 & 0,855 & 1,056 & 1,313 & 1,701 & 2,048 & 2,467 & 2,763 \\
    29 & 0,854 & 1,055 & 1,311 & 1,699 & 2,045 & 2,462 & 2,756 \\
    30 & 0,854 & 1,055 & 1,31 & 1,697 & 2,042 & 2,459 & 2,75 \\
    40 & 0,851 & 1,05 & 1,303 & 1,684 & 2,021 & 2,423 & 2,705 \\
    60 & 0,848 & 1,046 & 1,296 & 1,671 & 1,997 & 2,39 & 2,86 \\
    120 & 0,845 & 1,041 & 1,289 & 1,658 & 1,98 & 2,358 & 2,617 \\
    \hline
\end{tabular}
\end{center} 

\end{boxK}

\begin{boxK}{z-Verteilung}
    \section*{Standardnormalverteilung}
In der Tabelle findet sich die Fläche, die von einem bestimmten $z$‐Wert abgeschnitten wird. Dabei ist die erste Stelle hinter dem Komma des $z$‐Wertes in der linken Spalte zu finden und die zweite Stelle hinter dem Komma in der ersten Zeile. Ein $z$‐Wert von 1,23 schneidet beispielsweise eine Fläche von 0,8907 ab. Ein $z$‐Wert muss die gleiche oder mehr als die Fläche des jeweiligen Signifikanzniveaus abschneiden. Bei einem Alpha-Niveau von 5\% muss die Fläche also mindestens 0,95 betragen, bei einem Alpha-Niveau von 1\% 
mindestens 0,99.

\vspace{1 cm}
\begin{center}
\begin{tabular}{|c|c|c|c|c|c|c|c|c|c|c|}
    \hline
    z & 0 & 0,01 & 0,02 & 0,03 & 0,04 & 0,05 & 0,06 & 0,07 & 0,08 & 0,09 \\
    \hline
    0 & 0,5 & 0,504 & 0,508 & 0,512 & 0,516 & 0,5199 & 0,5239 & 0,5279 & 0,5319 & 0,5359 \\
    0,1 & 0,5398 & 0,5438 & 0,5478 & 0,5517 & 0,5557 & 0,5596 & 0,5636 & 0,5675 & 0,5714 & 0,5753 \\
    0,2 & 0,5793 & 0,5832 & 0,5871 & 0,591 & 0,5948 & 0,5987 & 0,6026 & 0,6064 & 0,6103 & 0,6141 \\
    0,3 & 0,6179 & 0,6217 & 0,6255 & 0,6293 & 0,6331 & 0,6368 & 0,6406 & 0,6443 & 0,648 & 0,6517 \\
    0,4 & 0,6554 & 0,6591 & 0,6628 & 0,6664 & 0,67 & 0,6736 & 0,6772 & 0,6808 & 0,6844 & 0,6879 \\
    0,5 & 0,6915 & 0,695 & 0,6985 & 0,7019 & 0,7054 & 0,7088 & 0,7123 & 0,7157 & 0,719 & 0,7224 \\
    0,6 & 0,7257 & 0,7291 & 0,7324 & 0,7357 & 0,7389 & 0,7422 & 0,7454 & 0,7486 & 0,7517 & 0,7549 \\
    0,7 & 0,758 & 0,7611 & 0,7642 & 0,7673 & 0,7704 & 0,7734 & 0,7764 & 0,7794 & 0,7823 & 0,7852 \\
    0,8 & 0,7881 & 0,791 & 0,7939 & 0,7967 & 0,7995 & 0,8023 & 0,8051 & 0,8079 & 0,8106 & 0,8133\\
    0,9 & 0,8158 & 0,8186 & 0,8212 & 0,8238 & 0,8264 & 0,8289 & 0,8315 & 0,834 & 0,8365 & 0,8398\\
    1 & 0,8413 & 0,8438 & 0,8461 & 0,8485 & 0,8508 & 0,8531 & 0,8554 & 0,8577 & 0,8599 & 0,8621 \\
    1,1 & 0,8643 & 0,8665 & 0,8686 & 0,8708 & 0,8729 & 0,8749 & 0,877 & 0,879 & 0,881 & 0,883 \\
    1,2 & 0,8849 & 0,8869 & 0,8888 & 0,8907 & 0,8925 & 0,8944 & 0,8962 & 0,898 & 0,8997 & 0,9015\\
    1,3 & 0,9032 & 0,9049 & 0,9066 & 0,9082 & 0,9099 & 0,9115 & 0,9131 & 0,9147 & 0,9162 & 0,9177\\
    1,4 & 0,9192 & 0,9207 & 0,9222 & 0,9236 & 0,9251 & 0,9265 & 0,9279 & 0,9292 & 0,9306 & 0,9319\\
    1,5 & 0,9332 & 0,9345 & 0,9357 & 0,937 & 0,9382 & 0,9304 & 0,9406 & 0,9418 & 0,9429 & 0,9441\\
    1,6 & 0,9452 & 0,9463 & 0,9474 & 0,9484 & 0,9495 & 0,9505 & 0,9515 & 0,9525 & 0,9535 & 0,9545\\
    1,7 & 0,9554 & 0,9564 & 0,9573 & 0,9582 & 0,9591 & 0,9599 & 0,9608 & 0,9616 & 0,9625 & 0,9633 \\
    1,8 & 0,9641 & 0,9649 & 0,9656 & 0,9664 & 0,9671 & 0,9678 & 0,9686 & 0,9693 & 0,9699 & 0,9706 \\
    1,9 & 0,9713 & 0,9719 & 0,9726 & 0,9723 & 0,9738 & 0,9744 & 0,975 & 0,9756 & 0,9761 & 0,9767\\
    2 & 0,9772 & 0,9778 & 0,9783 & 0,9788 & 0,9793 & 0,9798 & 0,9803 & 0,9808 & 0,9812 & 0,9817\\
    2,1 & 0,9821 & 0,9826 & 0,983 & 0,9834 & 0,9838 & 0,9842 & 0,9846 & 0,985 & 0,9854 & 0,9857\\
    2,2 & 0,9861 & 0,9864 & 0,9868 & 0,9871 & 0,9875 & 0,9878 & 0,9881 & 0,9884 & 0,9887 & 0,989\\
    2,3 & 0,9893 & 0,9896 & 0,9898 & 0,9901 & 0,9904 & 0,9906 & 0,9909 & 0,9911 & 0,9913 & 0,9916\\
    2,4 & 0,9918 & 0,992 & 0,9922 & 0,9925 & 0,9927 & 0,9929 & 0,9931 & 0,9932 & 0,9934 & 0,9936\\
    2,5 & 0,9938 & 0,994 & 0,9941 & 0,9943 & 0,9945 & 0,9946 & 0,9948 & 0,9949 & 0,9951 & 0,9952\\
    2,6 & 0,9953 & 0,9955 & 0,9956 & 0,9957 & 0,9959 & 0,996 & 0,9961 & 0,9962 & 0,9963 & 0,9964\\
    2,7 & 0,9965 & 0,9966 & 0,9967 & 0,9968 & 0,9969 & 0,997 & 0,9971 & 0,9972 & 0,9973 & 0,9974\\
    2,8 & 0,9974 & 0,9975 & 0,9976 & 0,9977 & 0,9977 & 0,9978 & 0,9979 & 0,9979 & 0,998 & 0,9981\\
    2,9 & 0,9981 & 0,9982 & 0,9982 & 0,9983 & 0,9984 & 0,9984 & 0,9985 & 0,9985 & 0,9986 & 0,9986 \\
    3,0 & 0,9987 & 0,9987 & 0,9987 & 0,9988 & 0,9988 & 0,9989 & 0,9989 & 0,9989 & 0,999 & 0,999\\
    \hline
\end{tabular}
\end{center} 
\end{boxK}

\begin{boxK}{F-Verteilung}
In der Tabelle finden sich die kritischen $F$‐Werte. Beim $F$‐Test sind sowohl Zähler-  als auch Nennerfreiheitsgrade relevant, die sich in der linken Spalte und der ersten Zeile finden. Die zweite Spalte enthält die Fläche, die mit einem bestimmten Alpha-Niveau assoziiert ist, zum Beispiel 0,95 mit einem Alpha-Niveau von 5\%. Der empirische $F$‐Wert muss gleich groß oder größer sein als der kritische $F$‐Wert in der Tabelle, um auf dem entsprechenden Niveau signifikant zu sein.

\vspace{1 cm}
\begin{center}
\begin{tabular}{|c|c|c|c|c|c|c|c|c|c|c|c|c|c|}
    \hline
    Nenner df & Fläche & 1 & 2 & 3 & 4 & 5 & 6 & 7 & 8 & 9 & 10 & 11 & 12\\
    \hline
     & 0,75 & 5,83 & 7,5 & 8,2 & 8,58 & 8,82 & 8,98 & 9,1 & 9,19 & 9,26 & 9,32 & 9,36 & 9,41\\
    1 & 0,9 & 39,9 & 49,5 & 53,6 & 55,8 & 57,2 & 58,2 & 58,9 & 59,4 & 59,9 & 60,2 & 60,5 & 60,7\\
     & 0,95 & 161 & 200 & 216 & 225 & 230 & 234 & 237 & 239 & 241 & 242 & 243 & 244\\
     \hline
     & 0,75 & 2,57 & 3 & 3,15 & 3,23 & 3,28 & 3,31 & 3,34 & 3,35 & 3,37 & 3,38 & 3,39 & 3,39\\
    2 & 0,9 & 8,53 & 9 & 9,16 & 9,24 & 9,29 & 9,33 & 9,35 & 9,37 & 9,38 & 9,39 & 9,4 & 9,41\\
     & 0,95 & 18,5 & 19 & 19,2 & 19,2 & 19,3 & 19,3 & 19,4 & 19,4 & 19,4 & 19,4 & 19,4 & 19,4\\
     & 0,99 & 98,5 & 99 & 99,2 & 99,2 & 99,3 & 99,3 & 99,4 & 99,4 & 99,4 & 99,4 & 99,4 & 99,4\\
     \hline
     & 0,75 & 2,02 & 2,28 & 2,36 & 2,39 & 2,41 & 2,42 & 2,43 & 2,44 & 2,44 & 2,44 & 2,45 & 2,45\\
    3 & 0,9 & 5,54 & 5,46 & 5,39 & 5,34 & 5,31 & 5,28 & 5,27 & 5,25 & 5,24 & 5,23 & 5,22 & 5,22\\
     & 0,95 & 10,1 & 9,55 & 9,28 & 9,12 & 9,1 & 8,94 & 8,89 & 8,85 & 8,81 & 8,79 & 9,76 & 8,74\\
     & 0,99 & 34,1 & 30,8 & 29,5 & 28,7 & 28,2 & 27,9 & 27,7 & 27,5 & 27,3 & 27,2 & 27,1 & 27,1\\
     \hline
     & 0,75 & 1,81 & 2 & 2,05 & 2,06 & 2,07 & 2,08 & 2,08 & 2,08 & 2,08 & 2,08 & 2,08 & 2,08\\
    4 & 0,9 & 4,54 & 4,32 & 4,19 & 4,11 & 4,05 & 4,01 & 3,98 & 3,95 & 3,94 & 3,92 & 3,91 & 3,9\\
     & 0,95 & 7,71 & 6,94 & 6,59 & 6,39 & 6,26 & 6,16 & 6,09 & 6,04 & 6 & 5,96 & 5,94 & 5,91\\
     & 0,99 & 21,2 & 18 & 16,7 & 16 & 15,5 & 15,2 & 15 & 14,8 & 14,7 & 14,5 & 14,1 & 14,4\\
     \hline
     & 0,75 & 1,69 & 1,85 & 1,88 & 1,89 & 1,89 & 1,89 & 1,89 & 1,89 & 1,89 & 1,89 & 1,89 & 1,89\\
    5 & 0,9 & 4,06 & 3,78 & 3,62 & 3,52 & 3,45 & 3,4 & 3,37 & 3,34 & 3,32 & 3,3 & 3,28 & 3,27\\
     & 0,95 & 6,61 & 5,79 & 5,41 & 5,19 & 5,05 & 4,95 & 4,88 & 4,82 & 4,77 & 4,74 & 4,71 & 4,68\\
     & 0,99 & 16,3 & 13,3 & 12,1 & 11,4 & 11 & 10,7 & 10,5 & 10,3 & 10,2 & 10,1 & 9,96 & 9,89\\
     \hline
     & 0,75 & 1,62 & 1,76 & 1,78 & 1,79 & 1,79 & 1,78 & 1,78 & 1,77 & 1,77 & 1,77 & 1,77 & 1,77\\
    6 & 0,9 & 3,78 & 3,46 & 3,29 & 3,18 & 3,11 & 3,05 & 3,01 & 2,98 & 2,96 & 2,94 & 2,92 & 2,9\\
     & 0,95 & 5,99 & 5,14 & 4,76 & 4,53 & 4,39 & 4,28 & 4,21 & 4,15 & 4,1 & 4,06 & 4,03 & 4\\
     & 0,99 & 13,7 & 10,9 & 9,78 & 9,15 & 8,75 & 8,47 & 8,26 & 8,1 & 7,98 & 7,87 & 7,79 & 7,72\\
     \hline
     & 0,75 & 1,57 & 1,7 & 1,72 & 1,72 & 1,71 & 1,71 & 1,7 & 1,7 & 1,69 & 1,69 & 1,69 & 1,68\\
    7 & 0,9 & 3,59 & 3,26 & 3,07 & 2,96 & 2,88 & 2,83 & 2,78 & 2,75 & 2,72 & 2,7 & 2,68 & 2,67\\
     & 0,95 & 5,59 & 4,74 & 4,35 & 4,12 & 3,97 & 3,87 & 3,79 & 3,73 & 3,68 & 3,64 & 3,6 & 3,57\\
     & 0,99 & 12,2 & 9,55 & 8,45 & 7,85 & 7,46 & 7,19 & 6,99 & 6,84 & 6,72 & 6,62 & 6,54 & 6,47\\
     \hline
     & 0,75 & 1,54 & 1,66 & 1,67 & 1,66 & 1,66 & 1,65 & 1,64 & 1,64 & 1,64 & 1,63 & 1,63 & 1,62\\
    8 & 0,9 & 3,46 & 3,11 & 2,92 & 2,81 & 2,73 & 2,67 & 2,62 & 2,59 & 2,56 & 2,54 & 2,52 & 2,5\\
     & 0,95 & 5,32 & 4,46 & 4,07 & 3,84 & 3,69 & 3,58 & 3,5 & 3,44 & 3,39 & 3,35 & 3,31 & 3,28\\
     & 0,99 & 11,3 & 8,65 & 7,59 & 7,01 & 6,63 & 6,37 & 6,18 & 6,03 & 5,91 & 5,81 & 5,73 & 5,67\\
     \hline
     & 0,75 & 1,51 & 1,62 & 1,63 & 1,63 & 1,62 & 1,61 & 1,6 & 1,6 & 1,59 & 1,59 & 1,58 & 1,58\\
    9 & 0,9 & 3,36 & 3,01 & 2,81 & 2,69 & 2,61 & 2,55 & 2,51 & 2,47 & 2,44 & 2,42 & 2,4 & 2,38\\
     & 0,95 & 5,12 & 4,26 & 3,86 & 3,63 & 3,48 & 3,37 & 3,29 & 3,23 & 3,18 & 3,14 & 3,1 & 3,07\\
     & 0,99 & 10,6 & 8,02 & 6,99 & 6,42 & 6,06 & 5,8 & 5,61 & 5,47 & 5,35 & 5,26 & 5,18 & 5,11\\
     \hline
     & 0,75 & 1,49 & 1,6 & 1,6 & 1,59 & 1,59 & 1,58 & 1,57 & 1,56 & 1,56 & 1,55 & 1,55 & 1,54\\
    10 & 0,9 & 3,28 & 2,92 & 2,73 & 2,61 & 2,52 & 2,46 & 2,41 & 2,39 & 2,35 & 2,32 & 2,3 & 2,28\\
     & 0,95 & 4,96 & 4,1 & 3,71 & 3,48 & 3,33 & 3,22 & 3,14 & 3,07 & 3,02 & 2,98 & 2,94 & 2,91\\
     & 0,99 & 10 & 7,56 & 6,55 & 5,99 & 5,64 & 5,39 & 5,2 & 5,06 & 4,94 & 4,85 & 4,77 & 4,71\\
     \hline
\end{tabular}
\end{center}
\end{boxK}

\begin{boxK}{F-Verteilung II}
\begin{center}
\begin{tabular}{|c|c|c|c|c|c|c|c|c|c|c|c|c|c|}
    \hline
    Nenner df & Fläche & 1 & 2 & 3 & 4 & 5 & 6 & 7 & 8 & 9 & 10 & 11 & 12\\
    \hline
     & 0,75 & 1,47 & 1,58 & 1,58 & 1,57 & 1,56 & 1,55 & 1,54 & 1,53 & 1,53 & 1,52 & 1,52 & 1,51\\
    11 & 0,9 & 3,23 & 2,86 & 2,66 & 1,54 & 2,45 & 2,39 & 2,34 & 2,3 & 2,27 & 2,25 & 2,23 & 2,21\\
     & 0,95 & 4,84 & 3,98 & 3,59 & 3,36 & 3,2 & 3,09 & 3,01 & 2,95 & 2,9 & 2,85 & 2,82 & 2,79\\
     & 0,99 & 9,65 & 7,21 & 6,22 & 5,67 & 5,32 & 5,07 & 4,89 & 4,74 & 4,63 & 4,54 & 4,46 & 4,4\\
     \hline
     & 0,75 & 1,46 & 1,56 & 1,56 & 1,55 & 1,54 & 1,53 & 1,52 & 1,51 & 1,51 & 1,5 & 1,5 & 1,49\\
    12 & 0,9 & 3,18 & 2,81 & 2,61 & 2,48 & 2,39 & 2,33 & 2,28 & 2,24 & 2,21 & 2,19 & 2,17 & 2,15\\
     & 0,95 & 4,75 & 3,89 & 3,49 & 3,26 & 3,11 & 3 & 2,91 & 2,85 & 2,8 & 2,75 & 2,72 & 2,69\\
     & 0,99 & 9,33 & 6,93 & 5,95 & 5,41 & 5,06 & 4,82 & 4,64 & 4,5 & 4,39 & 4,3 & 4,22 & 4,16\\
     \hline
     & 0,75 & 1,45 & 1,54 & 1,54 & 1,53 & 1,52 & 1,51 & 1,5 & 1,49 & 1,49 & 1,48 & 1,47 & 1,47\\
    13 & 0,9 & 3,14 & 2,76 & 2,56 & 2,43 & 2,35 & 2,28 & 2,23 & 2,2 & 2,16 & 2,14 & 2,12 & 2,1\\
     & 0,95 & 4,67 & 3,81 & 3,41 & 3,18 & 3,03 & 2,92 & 2,83 & 2,77 & 2,71 & 2,67 & 2,63 & 2,6\\
     & 0,99 & 9,07 & 6,7 & 5,74 & 5,21 & 4,86 & 4,62 & 4,44 & 4,3 & 4,19 & 4,1 & 4,02 & 3,96\\
     \hline
     & 0,75 & 1,44 & 1,53 & 1,53 & 1,52 & 1,51 & 1,5 & 1,48 & 1,48 & 1,47 & 1,46 & 1,46 & 1,45\\
     14 & 0,9 & 3,1 & 2,73 & 2,52 & 2,39 & 2,31 & 2,24 & 2,19 & 2,15 & 2,12 & 2,1 & 2,08 & 2,05\\
      & 0,95 & 4,6 & 3,74 & 3,34 & 3,11 & 2,96 & 2,85 & 2,76 & 2,7 & 2,65 & 2,6 & 2,57 & 2,53\\
      & 0,99 & 8,86 & 6,51 & 5,56 & 5,04 & 4,69 & 4,46 & 4,28 & 4,14 & 4,03 & 3,94 & 3,86 & 3,8\\
      \hline
      & 0,75 & 1,43 & 1,52 & 1,52 & 1,51 & 1,49 & 1,48 & 1,47 & 1,46 & 1,46 & 1,45 & 1,44 & 1,44\\
    15 & 0,9 & 3,07 & 2,7 & 2,49 & 2,36 & 2,27 & 2,21 & 2,16 & 2,12 & 2,09 & 2,06 & 2,04 & 2,02\\
     & 0,95 & 4,54 & 3,68 & 3,29 & 3,06 & 2,9 & 2,79 & 2,71 & 2,64 & 2,59 & 2,54 & 2,51 & 2,48\\
     & 0,99 & 8,68 & 6,36 & 5,42 & 4,89 & 4,56 & 4,32 & 4,14 & 4 & 3,89 & 3,8 & 3,73 & 3,67\\
     \hline
     & 0,75 & 1,42 & 1,51 & 1,51 & 1,5 & 1,48 & 1,48 & 1,47 & 1,46 & 1,45 & 1,45 & 1,44 & 1,44\\
    16 & 0,9 & 3,05 & 2,67 & 2,46 & 2,33 & 2,24 & 2,18 & 2,13 & 2,09 & 2,06 & 2,03 & 2,01 & 1,99\\
     & 0,95 & 4,49 & 3,63 & 3,24 & 3,01 & 2,85 & 2,74 & 2,66 & 2,59 & 2,54 & 2,49 & 2,46 & 2,42\\
     & 0,99 & 8,53 & 6,23 & 5,29 & 4,77 & 4,44 & 4,2 & 4,03 & 3,89 & 3,78 & 3,69 & 3,62 & 3,55\\
     \hline
     & 0,75 & 1,42 & 1,51 & 1,5 & 1,49 & 1,47 & 1,46 & 1,45 & 1,44 & 1,43 & 1,43 & 1,42 & 1,41\\
    17 & 0,9 & 3,03 & 2,64 & 2,44 & 2,31 & 2,22 & 2,15 & 2,1 & 2,06 & 2,03 & 2 & 1,98 & 1,96\\
     & 0,95 & 4,45 & 3,59 & 3,2 & 2,96 & 2,81 & 2,7 & 2,61 & 2,55 & 2,49 & 2,45 & 2,41 & 2,38\\
     & 0,99 & 8,4 & 6,11 & 5,18 & 4,67 & 4,34 & 4,1 & 3,93 & 3,79 & 3,68 & 3,59 & 3,52 & 3,46\\
     \hline
     & 0,75 & 1,41 & 1,5 & 1,49 & 1,48 & 1,46 & 1,45 & 1,44 & 1,43 & 1,42 & 1,42 & 1,41 & 1,4\\
    18 & 0,9 & 3,01 & 2,62 & 2,42 & 2,29 & 2,2 & 2,13 & 2,08 & 2,04 & 2 & 1,98 & 1,96 & 1,93\\
     & 0,95 & 4,41 & 3,55 & 3,16 & 2,93 & 2,77 & 2,66 & 2,58 & 2,51 & 2,46 & 2,41 & 2,37 & 2,34\\
     & 0,99 & 8,29 & 6,01 & 5,09 & 4,58 & 5,24 & 4,01 & 3,84 & 3,71 & 3,6 & 3,51 & 3,43 & 3,37\\
     \hline
     & 0,75 & 1,41 & 1,49 & 1,49 & 1,47 & 1,46 & 1,44 & 1,43 & 1,42 & 1,41 & 1,41 & 1,4 & 1,4\\
    19 & 0,9 & 2,99 & 2,61 & 2,4 & 2,27 & 2,18 & 2,11 & 2,06 & 2,02 & 1,98 & 1,96 & 1,94 & 1,91\\
     & 0,95 & 4,38 & 3,52 & 3,13 & 2,9 & 2,74 & 2,63 & 2,54 & 2,48 & 2,42 & 2,38 & 2,34 & 2,31\\
     & 0,99 & 8,18 & 5,93 & 5,01 & 4,5 & 4,17 & 3,94 & 3,77 & 3,63 & 3,52 & 3,43 & 3,36 & 3,3\\
     \hline
     & 0,75 & 1,4 & 1,49 & 1,48 & 1,46 & 1,45 & 1,44 & 1,42 & 1,42 & 1,41 & 1,4 & 1,39 & 1,39\\
    20 & 0,9 & 2,97 & 2,59 & 2,38 & 2,25 & 2,16 & 2,09 & 2,04 & 2 & 1,96 & 1,94 & 1,92 & 1,89\\
     & 0,95 & 4,35 & 3,49 & 3,1 & 2,87 & 2,71 & 2,6 & 2,51 & 2,45 & 2,39 & 2,35 & 2,31 & 2,28\\
     & 0,99 & 8,1 & 5,85 & 4,94 & 4,43 & 4,1 & 3,87 & 3,7 & 3,56 & 3,46 & 3,37 & 3,29 & 3,23\\
     \hline
     & 0,75 & 1,4 & 1,48 & 1,47 & 1,45 & 1,44 & 1,42 & 1,41 & 1,4 & 1,39 & 1,39 & 1,38 & 1,37\\
    22 & 0,9 & 2,95 & 2,56 & 2,35 & 2,22 & 2,13 & 2,06 & 2,01 & 1,97 & 1,93 & 1,9 & 1,88 & 1,86\\
     & 0,95 & 4,3 & 3,44 & 3,05 & 2,82 & 2,66 & 2,55 & 2,46 & 2,4 & 2,34 & 2,3 & 2,26 & 2,23\\
     & 0,99 & 7,95 & 5,72 & 4,82 & 4,31 & 3,99 & 3,76 & 3,59 & 3,45 & 3,35 & 3,26 & 3,18 & 3,12\\
     \hline
\hline
\end{tabular}
\end{center} 
\end{boxK}


\begin{boxK}{F-Verteilung III}
\begin{center}
    \begin{tabular}{|c|c|c|c|c|c|c|c|c|c|c|c|c|c|}
    \hline
    Nenner df & Fläche & 1 & 2 & 3 & 4 & 5 & 6 & 7 & 8 & 9 & 10 & 11 & 12\\
    \hline
    & 0,75 & 1,39 & 1,47 & 1,46 & 1,44 & 1,43 & 1,41 & 1,4 & 1,39 & 1,38 & 1,38 & 1,37 & 1,36\\
    24 & 0,9 & 2,93 & 2,54 & 2,33 & 2,19 & 2,1 & 2,04 & 1,98 & 1,94 & 1,91 & 1,88 & 1,85 & 1,83\\
     & 0,95 & 4,26 & 3,4 & 3,01 & 2,78 & 2,62 & 2,51 & 2,42 & 2,36 & 2,3 & 2,25 & 2,21 & 2,18\\
     & 0,99 & 7,82 & 5,61 & 4,72 & 4,22 & 3,9 & 3,67 & 3,5 & 3,36 & 3,26 & 3,17 & 3,09 & 3,03\\
     \hline
     & 0,75 & 1,38 & 1,46 & 1,45 & 1,44 & 1,42 & 1,41 & 1,4 & 1,39 & 1,37 & 1,37 & 1,36 & 1,35\\
    26 & 0,9 & 2,91 & 2,52 & 2,31 & 2,17 & 2,08 & 2,01 & 1,96 & 1,92 & 1,88 & 1,86 & 1,84 & 1,81\\
     & 0,95 & 4,23 & 3,37 & 2,98 & 2,74 & 2,59 & 2,47 & 2,39 & 2,32 & 2,27 & 2,22 & 2,18 & 2,15\\
     & 0,99 & 7,72 & 5,53 & 4,64 & 4,14 & 3,82 & 3,59 & 3,42 & 3,29 & 3,18 & 3,09 & 3,02 & 2,96\\
     \hline
     & 0,75 & 1,38 & 1,46 & 1,45 & 1,43 & 1,41 & 1,4 & 1,39 & 1,38 & 1,37 & 1,36 & 1,35 & 1,34\\
    28 & 0,9 & 2,89 & 2,5 & 2,29 & 2,16 & 2,06 & 2 & 1,94 & 1,9 & 1,87 & 1,84 & 1,81 & 1,79\\
     & 0,95 & 4,2 & 3,34 & 2,95 & 2,71 & 2,56 & 2,45 & 2,36 & 2,29 & 2,24 & 2,19 & 2,15 & 2,12\\
     & 0,99 & 7,64 & 5,45 & 4,57 & 4,07 & 3,75 & 3,53 & 3,36 & 3,23 & 3,12 & 3,03 & 2,96 & 2,9\\
     \hline
     & 0,75 & 1,38 & 1,45 & 1,44 & 1,42 & 1,41 & 1,39 & 1,38 & 1,37 & 1,36 & 1,35 & 1,35 & 1,34\\
    30 & 0,9 & 2,88 & 2,49 & 2,28 & 2,14 & 2,05 & 1,98 & 1,93 & 1,88 & 1,85 & 1,82 & 1,79 & 1,77\\
     & 0,95 & 4,17 & 3,32 & 2,92 & 2,69 & 2,53 & 2,42 & 2,33 & 2,27 & 2,21 & 2,16 & 2,13 & 2,09\\
     & 0,99 & 7,56 & 5,39 & 4,51 & 4,02 & 3,7 & 3,47 & 3,3 & 3,17 & 3,07 & 2,98 & 2,91 & 2,84\\
     \hline
     & 0,75 & 1,36 & 1,44 & 1,42 & 1,4 & 1,39 & 1,37 & 1,36 & 1,35 & 1,34 & 1,33 & 1,32 & 1,31\\
    40 & 0,9 & 2,84 & 2,44 & 2,23 & 2,09 & 2 & 1,93 & 1,87 & 1,83 & 1,79 & 1,76 & 1,73 & 1,71\\
     & 0,95 & 4,08 & 3,23 & 2,84 & 2,61 & 2,45 & 2,34 & 2,25 & 2,18 & 2,12 & 2,08 & 2,04 & 2\\
     & 0,99 & 7,31 & 5,18 & 4,31 & 3,83 & 3,51 & 3,29 & 3,12 & 2,99 & 2,89 & 2,8 & 2,73 & 2,66\\
     \hline
     & 0,75 & 1,35 & 1,42 & 1,41 & 1,38 & 1,37 & 1,35 & 1,33 & 1,32 & 1,31 & 1,3 & 1,29 & 1,29\\
    60 & 0,9 & 2,79 & 2,39 & 2,18 & 2,04 & 1,95 & 1,87 & 1,82 & 1,77 & 1,74 & 1,71 & 1,58 & 1,66\\
     & 0,95 & 4 & 3,15 & 2,76 & 2,53 & 2,37 & 2,25 & 2,17 & 2,1 & 2,04 & 1,99 & 1,95 & 1,92\\
     & 0,99 & 7,08 & 4,98 & 4,13 & 3,65 & 3,34 & 3,12 & 2,95 & 2,82 & 2,72 & 2,63 & 2,56 & 2,5\\
     \hline
\end{tabular}
\end{center}
\end{boxK}


\begin{boxK}{$\chi^2$-Verteilung}
In der Tabelle finden sich die kritischen $\chi^2$-Werte. Das Signifikanzniveau wird durch die Fläche angegeben. Beim Testen auf dem 5\%‐Niveau beträgt die relevante Fläche 0,95. Der Test ist immer einseitig. Der empirische $\chi^2$‐Wert muss gleich groß oder größer sein als der kritische $\chi^2$‐Wert aus der Tabelle, um auf dem entsprechenden Niveau signifikant zu sein.

\vspace{1 cm}
\begin{center}
\begin{tabular}{|c|c|c|c|c|c|}
    \hline
    df & 0,900 & 0,950 & 0,975 & 0,990 & 0,995\\
    \hline
    1 & 2,706 & 3,841 & 5,024 & 6,635 & 7,879\\
    2 & 4,605 & 5,991 & 7,378 & 9,210 & 10,600\\
    3 & 6,251 & 7,815 & 9,348 & 11,350 & 12,840\\
    4 & 7,779 & 9,488 & 11,140 & 13,280 & 14,860\\
    5 & 9,236 & 11,070 & 12,830 & 15,090 & 16,750\\
    6 & 10,650 & 12,590 & 14,450 & 16,810 & 18,550\\
    7 & 12,020 & 14,070 & 16,010 & 18,480 & 20,280\\
    8 & 13,360 & 15,510 & 17,540 & 20,090 & 21,960\\
    9 & 14,680 & 16,920 & 19,020 & 21,670 & 23,590\\
    10 & 15,990 & 18,310 & 20,480 & 23,210 & 25,190\\
    11 & 17,280 & 19,680 & 21,920 & 24,730 & 26,760\\
    12 & 18,550 & 21,030 & 23,340 & 26,220 & 28,300\\
    13 & 19,810 & 22,360 & 24,740 & 27,690 & 29,820\\
    14 & 21,060 & 23,690 & 26,120 & 29,140 & 31,320\\
    15 & 22,310 & 25,000 & 27,490 & 30,580 & 32,800\\
    16 & 23,540 & 26,300 & 28,850 & 32,000 & 34,270\\
    17 & 24,770 & 27,590 & 30,190 & 33,410 & 35,720\\
    18 & 25,990 & 28,870 & 31,530 & 34,810 & 37,160\\
    19 & 27,200 & 30,140 & 32,850 & 36,190 & 38,580\\
    20 & 28,410 & 31,410 & 34,170 & 37,570 & 40,000\\
    21 & 29,620 & 32,670 & 35,480 & 38,930 & 41,400\\
    22 & 30,810 & 33,920 & 36,780 & 40,290 & 42,800\\
    23 & 32,010 & 35,170 & 38,080 & 41,540 & 44,180\\
    24 & 33,200 & 36,410 & 39,360 & 42,980 & 45,560\\
    25 & 34,380 & 37,650 & 40,650 & 44,310 & 46,930\\
    26 & 35,560 & 38,890 & 41,920 & 45,640 & 48,290\\
    27 & 36,740 & 40,110 & 43,200 & 46,960 & 49,650\\
    28 & 37,920 & 41,340 & 44,460 & 48,280 & 50,990\\
    29 & 39,090 & 42,560 & 45,720 & 49,590 & 52,340\\
    30 & 40,260 & 43,770 & 46,980 & 50,890 & 53,670\\
    40 & 51,810 & 55,760 & 59,340 & 63,090 & 66,770\\
    50 & 63,170 & 67,510 & 71,420 & 76,150 & 79,490\\
    60 & 74,400 & 79,080 & 83,300 & 88,380 & 91,950\\
    70 & 85,530 & 90,530 & 95,020 & 100,400 & 104,200\\
    80 & 96,580 & 101,900 & 106,600 & 112,300 & 116,300\\
    90 & 107,600 & 113,200 & 118,100 & 124,100 & 128,300\\
    100 & 118,500 & 124,300 & 129,600 & 135,800 & 140,200\\
    \hline
\end{tabular}
\end{center}
\end{boxK}

\begin{boxK}{Wilcoxon-Test für AS}
In der Tabelle finden sich die kritischen $T$‐Werte des Wilcoxon-Tests. Das Signifikanzniveau ist in der ersten Zeile für einseitiges und in der zweiten Zeile für zweiseitiges Testen angegeben. 0,05 steht beispielsweise für ein Alpha-Niveau von 5\%. In der linken Spalte steht die Stichprobengröße. Der empirische $T$‐Wert muss gleich groß oder kleiner sein als der kritische $T$‐Wert aus der Tabelle, um auf dem entsprechenden Niveau signifikant zu sein.

\vspace{1 cm}
\begin{center}
\begin{tabular}{|c|c|c|c|c|}
\hline
einseitiges $\alpha$ & 0,05 & 0,025 & 0,01 & 0,005 \\
zweiseitiges $\alpha$ & 0,1 & 0,03 & 0,02 & 0,01 \\
\hline
    n &-&-&-&- \\
    5 & 0 &-&-&-\\
    6 & 2 & 0 &-&- \\
    7 & 3 & 2 & 0 &- \\
    8 & 5 & 4 & 2 & 0 \\
    9 & 8 & 6 & 3 & 2 \\
    10 & 10 & 8 & 5 & 3 \\
    11 & 13 & 11 & 7 & 5 \\
    12 & 17 & 14 & 10 & 7 \\
    13 & 21 & 17 & 13 & 10 \\
    14 & 25 & 21 & 16 & 13 \\
    15 & 30 & 25 & 20 & 16 \\
    16 & 35 & 30 & 24 & 20 \\
    17 & 41 & 35 & 28 & 23 \\
    18 & 47 & 40 & 33 & 28 \\
    19 & 53 & 46 & 38 & 32 \\
    20 & 60 & 52 & 43 & 38 \\
    21 & 67 & 59 & 49 & 43 \\
    22 & 75 & 66 & 56 & 49 \\
    23 & 83 & 73 & 62 & 55 \\
    24 & 91 & 81 & 69 & 61 \\
    25 & 100 & 89 & 77 & 68 \\
\hline
\end{tabular}
\end{center} 
\end{boxK}

\vspace{5cm}
\begin{boxK}{U-Verteilung}
In der Tabelle finden sich die kritischen $U$‐Werte. Das Signifikanzniveau für einseitiges Testen steht in der vorletzten und dasjenige für zweiseitiges Testen in der letzten Spalte. In der ersten Spalte ist die Stichprobengröße der ersten und in der ersten Zeile die Stichprobengröße der zweiten Gruppe angegeben. Der empirische $U$‐Wert muss gleich groß oder kleiner sein als der kritische $U$‐Wert aus der Tabelle, um auf dem entsprechenden Niveau signifikant zu sein. Bei zwei Stichproben von je 10 Personen und einem einseitigen Test auf dem 1\%‐Niveau muss der $U$‐Wert also gleich oder kleiner 19 sein.
\end{boxK}
\vspace{1 cm}
N2 = 1-20
\begin{center}
\scriptsize
\begin{tabular}{|c|c|c|c|c|c|c|c|c|c|c|c|c|c|c|c|c|c|c|c|c|c|c|}
\hline
N1 & 1 & 2 & 3 & 4 & 5 & 6 & 7 & 8 & 9 & 10 & 11 & 12 & 13 & 14 & 15 & 16 & 17 & 18 & 19 & 20 & einsietig & zweiseitig\\
\hline
1 &&&& &&&& &&&& &&&& &&&& & 1\% & 2\% \\
    &&&& &&&& &&&& &&&& &&& 0 & 0 & 5\% & 10\% \\
    \hline
2 &&&&&&&&&&&&& 0 & 0 & 0 & 0 & 0 & 0 & 1 & 1 & 1\% & 2\% \\
    &&&&& 0 & 0 & 0 & 1 & 1 & 1 & 1 & 2 & 2 & 2 & 3 & 3 & 3 & 4 & 4 & 4 & 5\% & 10\% \\
    \hline
3 &&&&&&& 0 & 0 & 1 & 1 & 1 & 2 & 2 & 2 & 3 & 3 & 4 & 4 & 4 & 5 & 1\% & 2\% \\
    &&& 0 & 0 & 1 & 2 & 2 & 3 & 3 & 4 & 5 & 5 & 6 & 7 & 7 & 8 & 9 & 9 & 10 & 11 & 5\% & 10\% \\
    \hline
4 &&&&& 0 & 1 & 1 & 2 & 3 & 3 & 4 & 5 & 5 & 6 & 7 & 7 & 8 & 9 & 9 & 10 & 1\% & 2\% \\
    &&& 0 & 1 & 2 & 3 & 4 & 5 & 6 & 7 & 8 & 9 & 10 & 11 & 12 & 14 & 15 & 16 & 17 & 18 & 5\% & 10\% \\
    \hline
5 &&&& 0 & 1 & 2 & 3 & 4 & 5 & 6 & 7 & 8 & 9 & 10 & 11 & 12 & 13 & 14 & 15 & 16 & 1\% & 2\% \\
    && 0 & 1 & 2 & 4 & 5 & 6 & 8 & 9 & 11 & 12 & 13 & 15 & 16 & 18 & 19 & 20 & 22 & 23 & 25 & 5\% & 10\% \\
    \hline
6 &&&& 1 & 2 & 3 & 4 & 6 & 7 & 8 & 9 & 11 & 12 & 13 & 15 & 16 & 18 & 19 & 20 & 22 & 1\% & 2\% \\
    && 0 & 2 & 3 & 5 & 7 & 8 & 10 & 12 & 14 & 16 & 17 & 19 & 21 & 23 & 25 & 26 & 28 & 30 & 32 & 5\% & 10\% \\
    \hline
7 &&& 0 & 1 & 3 & 4 & 6 & 7 & 9 & 11 & 12 & 14 & 16 & 17 & 19 & 21 & 23 & 24 & 26 & 28 & 1\% & 2\% \\
    && 0 & 2 & 4 & 6 & 8 & 11 & 13 & 15 & 17 & 19 & 21 & 24 & 26 & 28 & 30 & 33 & 35 & 37 & 39 & 5\% & 10\% \\
    \hline
8 &&& 0 & 2 & 4 & 6 & 7 & 9 & 11 & 13 & 15 & 17 & 20 & 22 & 24 & 26 & 28 & 30 & 32 & 34 & 1\% & 2\% \\
    && 1 & 3 & 5 & 8 & 10 & 13 & 15 & 18 & 20 & 23 & 26 & 28 & 31 & 33 & 36 & 39 & 41 & 44 & 47 & 5\% & 10\% \\
    \hline
9 &&& 1 & 3 & 5 & 7 & 9 & 11 & 14 & 16 & 18 & 21 & 23 & 26 & 28 & 31 & 33 & 36 & 38 & 40 & 1\% & 2\% \\
    && 1 & 3 & 6 & 9 & 12 & 15 & 18 & 21 & 24 & 27 & 30 & 33 & 36 & 39 & 42 & 45 & 48 & 51 & 54 & 5\% & 10\% \\
    \hline
10 &&& 1 & 3 & 6 & 8 & 11 & 13 & 16 & 19 & 22 & 24 & 27 & 30 & 33 & 36 & 38 & 41 & 44 & 47 & 1\% & 2\% \\
    && 1 & 4 & 7 & 11 & 14 & 17 & 20 & 24 & 27 & 31 & 34 & 37 & 41 & 44 & 48 & 51 & 55 & 58 & 62 & 5\% & 10\% \\
    \hline
11 &&& 1 & 4 & 7 & 9 & 12 & 15 & 18 & 22 & 25 & 28 & 31 & 34 & 37 & 41 & 44 & 47 & 50 & 53 & 1\% & 2\% \\
    && 1 & 5 & 8 & 12 & 16 & 19 & 23 & 27 & 31 & 34 & 38 & 42 & 46 & 50 & 54 & 57 & 61 & 65 & 69 & 5\% & 10\% \\
    \hline
12 &&& 2 & 5 & 8 & 11 & 14 & 17 & 21 & 24 & 28 & 31 & 35 & 38 & 42 & 46 & 49 & 53 & 56 & 60 & 1\% & 2\% \\
    && 2 & 5 & 9 & 13 & 17 & 21 & 26 & 30 & 34 & 38 & 42 & 47 & 51 & 55 & 60 & 64 & 68 & 72 & 77 & 5\% & 10\% \\
    \hline
13 && 0 & 2 & 5 & 9 & 12 & 16 & 20 & 23 & 27 & 31 & 35 & 39 & 43 & 47 & 51 & 55 & 59 & 63 & 67 & 1\% & 2\% \\
    && 2 & 6 & 10 & 15 & 19 & 24 & 28 & 33 & 37 & 42 & 47 & 51 & 56 & 61 & 65 & 70 & 75 & 80 & 84 & 5\% & 10\% \\
    \hline
14 && 0 & 2 & 6 & 10 & 13 & 17 & 22 & 26 & 30 & 34 & 38 & 43 & 47 & 51 & 56 & 60 & 65 & 69 & 73 & 1\% & 2\% \\
    && 2 & 7 & 11 & 16 & 21 & 26 & 31 & 36 & 41 & 46 & 51 & 56 & 61 & 66 & 71 & 77 & 82 & 87 & 92 & 5\% & 10\% \\
    \hline
15 && 0 & 3 & 7 & 11 & 15 & 19 & 24 & 28 & 33 & 37 & 42 & 47 & 51 & 56 & 61 & 66 & 70 & 75 & 80 & 1\% & 2\% \\
    && 3 & 7 & 12 & 18 & 23 & 28 & 33 & 39 & 44 & 50 & 55 & 61 & 66 & 72 & 77 & 83 & 88 & 94 & 100 & 5\% & 10\% \\
    \hline
16 && 0 & 3 & 7 & 12 & 16 & 21 & 26 & 31 & 36 & 41 & 46 & 51 & 56 & 61 & 66 & 71 & 76 & 82 & 87 & 1\% & 2\% \\
    && 3 & 8 & 14 & 19 & 25 & 30 & 36 & 42 & 48 & 54 & 60 & 65 & 71 & 77 & 83 & 89 & 95 & 101 & 107 & 5\% & 10\% \\
    \hline
17 && 0 & 4 & 8 & 13 & 18 & 23 & 28 & 33 & 38 & 44 & 49 & 55 & 60 & 66 & 71 & 77 & 82 & 88 & 93 & 1\% & 2\% \\
    && 3 & 9 & 15 & 20 & 26 & 33 & 39 & 45 & 51 & 57 & 64 & 70 & 77 & 83 & 89 & 96 & 102 & 109 & 115 & 5\% & 10\% \\
    \hline
18 && 0 & 4 & 9 & 14 & 19 & 24 & 30 & 36 & 41 & 47 & 53 & 59 & 65 & 70 & 76 & 82 & 88 & 94 & 100 & 1\% & 2\% \\
    && 4 & 9 & 16 & 22 & 28 & 35 & 41 & 48 & 55 & 61 & 68 & 75 & 82 & 88 & 95 & 102 & 109 & 116 & 123 & 5\% & 10\% \\
    \hline
19 && 1 & 4 & 9 & 15 & 20 & 26 & 32 & 38 & 44 & 50 & 56 & 63 & 69 & 75 & 82 & 88 & 94 & 101 & 107 & 1\% & 2\% \\
    & 0 & 4 & 10 & 17 & 23 & 30 & 37 & 44 & 51 & 58 & 65 & 72 & 80 & 87 & 94 & 101 & 109 & 116 & 123 & 130 & 5\% & 10\% \\
    \hline
20 && 1 & 5 & 10 & 16 & 22 & 28 & 34 & 40 & 47 & 53 & 60 & 67 & 73 & 80 & 87 & 93 & 100 & 107 & 114 & 1\% & 2\% \\
    & 0 & 4 & 11 & 18 & 25 & 32 & 39 & 47 & 54 & 62 & 69 & 77 & 84 & 92 & 100 & 107 & 115 & 123 & 130 & 138 & 5\% & 10\%\\
    \hline
\end{tabular}
\end{center}
Zellen ohne Werte geben an, dass eine Ablehnung der Nullhypothese auf dem entsprechenden Signifikanzniveau nicht möglich ist.

\newpage
\begin{boxK}{Danksagung}
Herzlicher Dank zur Mitarbeit an dieser Formelsammlung geht an Peter Sedlmeier, Isabell Winkler, Friederike Brockhaus und Markus Burkhardt.
\end{boxK}
\begin{boxK}{Conjoint-Measurement-Theory}
  Messaxomie fürs Intervallskalenniveau sind teilweise prüfbar über ein Conjoint-Measurement-Experiment. Hierbei geht man davon aus, dass $P=f(A,X)$ wobei $f$ eine non-interaktive Funktion ist. Geprüft werden muss die einfache Aufhebung (single cancellation) und die dopplete Aufhebung (double cancellation). Die einfache Aufhebung besagt, dass alle Zeilen und Spalten der Conjoint-Measurement-Matrix die gleiche Ordnung haben müssen. Die Doppelaufhebung restringiert die möglichen Ordnungen der Matrix. Wenn alle Spalten und Zeilen bereits geordnet sind, lässt sich über folgende Visualisierung die Doppelaufhebung prüfen. Hierbei entsprechen die Pfeile, dem $\leq$-Zeichen. Dünne Pfeile zwischen den Zellen repräsentieren die Antezedenz-Ordnungsbeziehungen, und dicke Pfeile repräsentieren die Konsequenz-Ordnungsbeziehung. 
  
\tikzset{
pics/alwaysthesame/.style={code={%

    % Draw the grid
    \draw[very thin, gray] (0, 0) grid (3, 3);
    
    % Add row labels
    \foreach \x in {1, 2, 3} {
        \node at (\x-0.5, 3.25) {$x_{\x}$};  % x1, x2, x3 labels
    }

    % Add column labels
    \foreach \y in {1, 2, 3} {
        \node at (-0.25, 3.5-\y) {$a_\y$};  % a1, a2, a3 labels
    }
    

    }
}}

% 1
\begin{tikzpicture}
    \pic{alwaysthesame};
        % Draw error lines (arrows)
    \draw[->, thick] (1.5, 2.5) -- (0.5, 1.5);  % (1,2) to (2,1)
    \draw[->, ultra thick] (2.5, 2.5) -- (0.5, 0.5);  % (1,3) to (3,1)
    \draw[->, thick] (2.5, 1.5) -- (1.5, 0.5);  % (2,3) to (3,2)
\end{tikzpicture}
% 2
\begin{tikzpicture}
    \pic{alwaysthesame};
        % Draw error lines (arrows)
    \draw[->, thick] (1.5, 2.5) -- (0.5, 1.5);  % (1,2) to (2,1)
    \draw[->, thick] (.5, .5) -- (2.5, 2.5);  % (1,3) to (3,1)
    \draw[->, ultra thick] (1.5, 0.5) -- (2.5, 1.5);  % (2,3) to (3,2)
\end{tikzpicture}
% 3
\begin{tikzpicture}
    \pic{alwaysthesame};
        % Draw error lines (arrows)
    \draw[->, ultra thick] (0.5, 1.5) -- (1.5, 2.5);  % (1,2) to (2,1)
    \draw[->, thick] (.5, .5) -- (2.5, 2.5);  % (1,3) to (3,1)
    \draw[->, thick] (2.5, 1.5) -- (1.5, 0.5);  % (2,3) to (3,2)
\end{tikzpicture}

% 4
\begin{tikzpicture}
    \pic{alwaysthesame};
        % Draw error lines (arrows)
    \draw[->, thick] (0.5, 1.5) -- (1.5, 2.5);  % (1,2) to (2,1)
    \draw[->, ultra thick] (.5, .5) -- (2.5, 2.5);  % (1,3) to (3,1)
    \draw[->, thick] (1.5, 0.5) -- (2.5, 1.5);  % (2,3) to (3,2)
\end{tikzpicture}
%% 5
%
\begin{tikzpicture}
    \pic{alwaysthesame};
        % Draw error lines (arrows)
    \draw[->, thick] (0.5, 1.5) -- (1.5, 2.5);  % (1,2) to (2,1)
    \draw[->, thick] (2.5, 2.5) -- (0.5, 0.5);  % (1,3) to (3,1)
    \draw[->, ultra thick] (2.5, 1.5) -- (1.5, 0.5);  % (2,3) to (3,2)
\end{tikzpicture}
% 6
\begin{tikzpicture}
    \pic{alwaysthesame};
        % Draw error lines (arrows)
    \draw[->, ultra thick] (1.5, 2.5) -- (0.5, 1.5);  % (1,2) to (2,1)
    \draw[->, thick] (2.5, 2.5) -- (0.5, 0.5);  % (1,3) to (3,1)
    \draw[->, thick] (1.5, 0.5) -- (2.5, 1.5);  % (2,3) to (3,2)
\end{tikzpicture}
\end{boxK}

\begin{boxK}{R Verteilungsfunktionen}
% Tabelle 1: Verteilungsfunktionen und Parameter
%\begin{table}[h!]

\begin{tabular}{@{}rll@{}}
\toprule
\textbf{Funktion} & \textbf{Verteilung}  & \textbf{Parameter} \\ \midrule
\texttt{dnorm}      & Normalverteilung     & \texttt{x}, \texttt{mean}, \texttt{sd} \\ 
\texttt{pnorm}      & Normalverteilung     & \texttt{q}, \texttt{mean}, \texttt{sd}, \texttt{lower.tail}, \texttt{log.p} \\ 
\texttt{qnorm}      & Normalverteilung     & \texttt{p}, \texttt{mean}, \texttt{sd}, \texttt{lower.tail}, \texttt{log.p} \\ 
\texttt{rnorm}      & Normalverteilung     & \texttt{n}, \texttt{mean}, \texttt{sd} \\ 
\texttt{dbinom}     & Binomialverteilung   & \texttt{x}, \texttt{size}, \texttt{prob} \\ 
\texttt{pbinom}     & Binomialverteilung   & \texttt{q}, \texttt{size}, \texttt{prob}, \texttt{lower.tail}, \texttt{log.p} \\ 
\texttt{qbinom}     & Binomialverteilung   & \texttt{p}, \texttt{size}, \texttt{prob}, \texttt{lower.tail}, \texttt{log.p} \\ 
\texttt{rbinom}     & Binomialverteilung   & \texttt{n}, \texttt{size}, \texttt{prob} \\ 
\texttt{dt}         & t-Verteilung         & \texttt{x}, \texttt{df}, \texttt{ncp} \\ 
\texttt{pt}         & t-Verteilung         & \texttt{q}, \texttt{df}, \texttt{ncp}, \texttt{lower.tail}, \texttt{log.p} \\ 
\texttt{qt}         & t-Verteilung         & \texttt{p}, \texttt{df}, \texttt{ncp}, \texttt{lower.tail}, \texttt{log.p} \\ 
\texttt{rt}         & t-Verteilung         & \texttt{n}, \texttt{df}, \texttt{ncp} \\ 
\texttt{df}         & F-Verteilung         & \texttt{x}, \texttt{df1}, \texttt{df2}, \texttt{ncp} \\ 
\texttt{pf}         & F-Verteilung         & \texttt{q}, \texttt{df1}, \texttt{df2}, \texttt{ncp}, \texttt{lower.tail}, \texttt{log.p} \\ 
\texttt{qf}         & F-Verteilung         & \texttt{p}, \texttt{df1}, \texttt{df2}, \texttt{ncp}, \texttt{lower.tail}, \texttt{log.p} \\ 
\texttt{rf}         & F-Verteilung         & \texttt{n}, \texttt{df1}, \texttt{df2}, \texttt{ncp} \\ 
\bottomrule
\end{tabular}
%\label{tab:verteilungs-funktionen}
%\end{table}

% Tabelle 2: Erklärung der Präfixe (r, q, d, p)
%\begin{table}[h!]
\vspace{0.5cm}

\begin{tabular}{@{}rl@{}}
\toprule
\textbf{Präfix} & \textbf{Erklärung} \\ \midrule
\texttt{d}      & Dichtefunktion: Berechnet die Wahrscheinlichkeitsdichte (z. B. \(f(x)\)) \\ 
\texttt{p}      & Verteilungsfunktion: Berechnet die kumulierte Wahrscheinlichkeit (z. B. \(P(X \leq x)\)) \\ 
\texttt{q}      & Quantilfunktion: Berechnet den Wert \(x\), sodass \(P(X \leq x) = p\) \\ 
\texttt{r}      & Zufallszahlengenerator: Generiert Zufallswerte aus der Verteilung \\ 
\bottomrule
\end{tabular}
\label{tab:praefix-erklaerung}

\vspace{0.5cm}
\begin{tabular}{@{}rl@{}}
\toprule
\textbf{Parameter}  & \textbf{Erklärung} \\ \midrule
\texttt{x}          & Werte, an denen die Dichte berechnet wird \\ 
\texttt{q}          & Quantile \\ 
\texttt{p}          & Wahrscheinlichkeiten \\ 
\texttt{n}          & Anzahl der zu generierenden Zufallswerte \\ 
\texttt{mean}       & Mittelwert der Verteilung \\ 
\texttt{sd}         & Standardabweichung \\ 
\texttt{size}       & Anzahl der Versuche \\ 
\texttt{prob}       & Erfolgswahrscheinlichkeit \\ 
\texttt{df}         & Freiheitsgrade \\ 
\texttt{df1}        & Freiheitsgrade des Zählers (F-Verteilung) \\ 
\texttt{df2}        & Freiheitsgrade des Nenners (F-Verteilung) \\ 
\texttt{ncp}        & Nichtzentralitätsparameter (optional) \\ 
\texttt{lower.tail} & Logisch; wenn TRUE (Standard), werden Wahrscheinlichkeiten \( P[X \leq x] \) berechnet \\ 
\texttt{log.p}      & Logisch; wenn TRUE, werden Wahrscheinlichkeiten als Log-Werte ausgegeben \\ 
\bottomrule
\end{tabular}
\end{boxK}


\begin{boxK}{Literatur}

Cohen, J. (1962). The statistical power of abnormal-social psychological research: A review. \textit{The Journal of Abnormal and Social Psychology}, 65(3), 146. \url{https://doi.org/10.1037/h0045186}
\vspace{0.25cm}

Sedlmeier, P., \& Renkewitz, F. (2018). Forschungsmethoden und Statistik für Psychologen und Sozialwissenschaftler: Pearson
\end{boxK}

\end{document}
