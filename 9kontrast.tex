
\begin{boxK}{Kontrastanalyse US}
\begin{multicols}{2}

$$F_{\mathrm{Kontrast}}=\frac{\hat{\sigma}^2_{\mathrm{Kontrast}}}{\hat{\sigma}^2_{\mathrm{inn}}}$$ 
\begin{center}
    oder
\end{center}
$$F=t^2$$
$$|t_{\mathrm{Kontrast}}|=\sqrt{F_{\mathrm{Kontrast}}}$$
\subsection*{Populationsvarianz}

\begin{align*}
\hat{\sigma}^2_{\mathrm{Kontrast}}&=\frac{\mathrm{QS}_{\mathrm{Kontrast}}}{\mathrm{df}_{\mathrm{Kontrast}}}\\
\hat{\sigma}^2_{\mathrm{inn}}&=\frac{\mathrm{QS}_\mathrm{inn}}{\mathrm{df}_\mathrm{inn}}\\
\mathrm{oder}\hspace{4mm}\hat{\sigma}^2_{\mathrm{inn}}&=\frac{\sum\limits_{j=1}^k \hat{\sigma}_j^2}{k}\\
\end{align*}

\subsection*{Quadratsummen}
$$\mathrm{QS}_{\mathrm{Kontrast}}=\frac{\left(\sum\limits_{i=1}^k \lambda_i\bar{x}_i\right)^2}{\sum\limits_{i=1}^k \frac{\lambda_i^2}{n_i}}$$
\subsection*{Freiheitsgrade}
$$\mathrm{df}_{\mathrm{Kontrast}}=1$$
$$\mathrm{df}_{\mathrm{inn}}=N-k$$

\subsection*{Notation}
\begin{tabular}{rl}
$k$: &Gruppe \\
$n_i$: &Anzahl der Messwerte pro Bedingung i\\
$n_j$: &Anzahl der Messwerte pro Bedingung j
\end{tabular}
\end{multicols}
\end{boxK}

\begin{boxK}{Kontrastanalyse für AS}
\begin{multicols}{2}
(ohne Subgruppen)
\begin{align*}
t_{\mathrm{Kontrast}} &= \frac{\bar{L}}{\sqrt{\frac{\hat{\sigma}^2_{L}}{n}}} \\
L_i &= \sum_{j=1}^k (x_{ij} \cdot \lambda_{j}) \\
\hat{\sigma}^2_{L} &= \frac{1}{n-1} \sum_{i=1}^n (L_i - \bar{L})^2 \\
\mathrm{df}_{\mathrm{Kontrast}} &= n-1
\end{align*}

\subsection*{Notation}
\begin{tabular}{rl}
$\bar{L}$: &Mittelwert $L-$Werte
\\$\hat{\sigma}^2_i$: &geschätzte Populationsvarianz der Gruppe i
\\$n$: &Anzahl der Objekte/Personen, \\
&für die mehrere Messungen durchgeführt
\end{tabular}
\end{multicols}
\end{boxK}
