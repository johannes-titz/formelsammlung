\begin{boxK}{Lagemaße}

\begin{multicols}{2}

\subsection*{Modalwert}

Der Modalwert ist der häufigste Wert aller Messwerte. Es kann mehrere Modal-Werte geben.


\subsection*{Mittelwert}
Auch bezeichnet als arithmetisches Mittel, für den Mittelwert der Population wird der Buchstabe $\mu$ verwendet.
\begin{equation*}
\bar{x}=\frac{1}{n}\sum\limits_{i=1}^n x_i
\end{equation*}

\subsection*{Median}

Der Median ist der Wert in der Mitte aller in einer Rangreihe geordneten Messwerte.

\begin{equation*}
\mathrm{Tiefe}_{\mathrm{Median}}=\frac{n+1}{2}
\end{equation*}

\subsection*{Quartile}
Die Werte müssen in eine Rangreihe gebracht werden. Für das untere Quartil wird die Tiefe von unten gezählt, fürs obere Quartil von oben. Quartile werden manchmal auch als die Quantile $Q_{25}, Q_{75}$ bezeichnet.
\begin{equation*}
\mathrm{Tiefe}_{\mathrm{Quartil}}=\frac{\lfloor\mathrm{Tiefe}_{\mathrm{Median}}\rfloor + 1}{2}
\end{equation*}

\subsection*{Zäune bei Boxplots}
Die Zäune werden benötigt um die Whiskers zu bestimmen. Die Whiskers sind durch tatsächlich vorkommende Werte repräsentiert. Von den Zäunen aus geht man in Richtung Box, bis man den ersten vorkommenden Wert findet.

\begin{align*}
  \mathrm{Zaun}_{\mathrm{unten}}&= Q_{25}-1,5 \cdot \mathrm{IQA}\\
  \mathrm{Zaun}_{\mathrm{oben}} &= Q_{75}+1,5 \cdot \mathrm{IQA}
\end{align*}


\end{multicols}
\end{boxK}

