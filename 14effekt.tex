\begin{boxK}{Effektgrößen aus Signifikanztests}
\begin{multicols}{2}
\subsection*{Konventionen nach Cohen}
"These values are necessarily somewhat arbitrary, but were chosen so as to seem reasonable. The reader can render his own judgment as to their reasonableness."
\vspace{0.5cm} (Cohen, 1962)

\begin{tabular}{lccc}
    \toprule
    & \multicolumn{1}{r}{d/g} & \multicolumn{1}{r}{r/$w$/$\phi$} & \multicolumn{1}{r}{$\eta^2$} \\
    \midrule
    klein & $\pm 0.2$ & $\pm 0.1$ & $0.01$ \\
    mittel & $\pm 0.5$ & $\pm 0.3$ & $0.06$ \\
    groß & $\pm 0.8$ & $\pm 0.5$ & $0.14$ \\
    \bottomrule
\end{tabular}
\vspace{1cm}

\subsection*{$t$-Test: Einstichprobenfall}
$$g=\frac{t}{\sqrt{n}}$$\\
$$d=\frac{t}{\sqrt{\mathrm{df}}}$$\\

\subsection*{$t$-Test: US}
$$g=t_\mathrm{US}\cdot\sqrt{\frac{n_\mathrm{A}+n_\mathrm{B}}{n_\mathrm{A}\cdot n_\mathrm{B}}}$$
%wenn $n_\mathrm{A}=n_\mathrm{B}$: 
$$d=t_\mathrm{US}\cdot\frac{n_\mathrm{A}+n_\mathrm{B}}{\sqrt{\mathrm{df}}\cdot\sqrt{n_\mathrm{A}\cdot n_\mathrm{B}}}$$
$$r=\sqrt{\frac{t^2_\mathrm{US}}{t^2_\mathrm{US}+\mathrm{df}}}$$
Letzteres gilt auch für $F$, wenn $\mathrm{df}_{\mathrm{zw}}$ 1 sind, da in diesem Fall $F=t^2$.\\

Vereinfachung, wenn $n_\mathrm{A}=n_\mathrm{B}$: 
$$d=\frac{2\cdot t_\mathrm{US}}{\sqrt{\mathrm{df}}}$$
$$g=\frac{2\cdot t_\mathrm{US}}{\sqrt{n}}$$


\subsection*{$t$-Test: AS}
$$g=\frac{t_\mathrm{AS}}{\sqrt{n}}$$
$$d=\frac{t_\mathrm{AS}}{\sqrt{\mathrm{df}}}$$


\subsection*{Kontrastanalyse US}
aus F-Test
$$r_{\mathrm{effect size}}=\sqrt{\frac{F_{\mathrm{Kontrast}}}{F_{\mathrm{zw}} \cdot {\mathrm{df}}_{\mathrm{zw}}+{\mathrm{df}}_{\mathrm{inn}}}}$$

$$r_{\mathrm{contrast}} = \sqrt{\frac{F_{\mathrm{Kontrast}}}{F_{\mathrm{Kontrast}}+\mathrm{df_{\mathrm{inn}}}}} = \sqrt{\frac{t^2_{\mathrm{Kontrast}}}{t^2_{\mathrm{Kontrast}}+\mathrm{df}}} $$
\subsection*{Kontrastanalyse AS}
aus t-Test
$$g=\frac{t}{\sqrt{n}}$$

\subsection*{$\chi^2$-Tests}
$$w=\sqrt{\sum\limits_{i=1}^k\frac{(P_{b,i}-P_{e,\;i})^2}{P_{e,i}}}$$
aus Ergebnis $\chi^2$-Test:
$$w=\sqrt{\frac{\chi^2}{N}}$$
aus 2 dichotomen Variablen
$$w=\sqrt{\frac{\chi^2}{N}}=\phi$$


\subsection*{ANOVA}

US
\begin{align*}
\eta^2&=\frac{F \cdot \mathrm{df}_{\mathrm{zw}}}{F \cdot \mathrm{df}_{\mathrm{zw}}+\mathrm{df}_\mathrm{inn}}\\
\eta^2&=\frac{\mathrm{QS}_{\mathrm{zw}}}{\mathrm{QS}_{\mathrm{ges}}}\\
\end{align*}
AS
\begin{align*}
\eta^2&=\frac{\mathrm{QS}_{\mathrm{UV}}}{\mathrm{QS}_{\mathrm{ges}}}\\
\eta^2_p&=\frac{\mathrm{QS}_{\mathrm{UV}}}{\mathrm{QS}_{\mathrm{UV}}+\mathrm{QS}_{\mathrm{res}}}\\
\end{align*}
mehrfaktoriell
\begin{align*}
\eta^2=\frac{\mathrm{QS}_{\mathrm{Effekt}}}{\mathrm{QS}_{\mathrm{ges}}}\\
\eta^2_p=\frac{\mathrm{QS}_{\mathrm{Effekt}}}{\mathrm{QS}_{\mathrm{Effekt}}+\mathrm{QS}_{\mathrm{inn}}}\\
\end{align*}

\end{multicols}
\end{boxK}

\begin{boxK}{Effektgrößen aus Rohwerten}
\begin{multicols}{3}

\begin{align*}
d &= \frac{\bar{x}-c}{s} \\
d &= \frac{\bar{x}_A-\bar{x}_B}{s_\mathrm{AB}} \\
g &= \frac{\bar{x}-c}{\hat{\sigma}} \\
g &= \frac{\bar{x}_A-\bar{x}_B}{\hat{\sigma}_\mathrm{AB}}\\
\end{align*}

\begin{math}
\begin{aligned}
s_\mathrm{AB} &= \sqrt{\frac{s_A^2n_A+s_B^2n_B}{n_A+n_B}}\\
\hat{\sigma}_\mathrm{AB} &= \sqrt{\frac{(n_A-1)\cdot \hat{\sigma}_A^2 + (n_B-1) \cdot \hat{\sigma}_B^2}{n_A+n_B-2}}
\end{aligned}
\end{math}

$$r_{\mathrm{effect size}}=\frac{\frac{1}{n}\sum\limits_i^n(x_i-\bar{x})\cdot(\lambda_i-\bar{\lambda})}{s_x \cdot s_{\lambda}}$$
$$r_{\mathrm{contrast}}= \frac{\mathrm{QS_{Kontrast}}}{\mathrm{QS_{Kontrast}}+\mathrm{QS_{Nicht-Kontrast}}}$$

$$r^2_{\mathrm{alerting}} = \frac{\mathrm{QS_{Kontrast}}}{\mathrm{QS_{zw}}}$$

$$g=\frac{\bar{L}}{\hat{\sigma}_{L}}$$

\end{multicols}
\end{boxK}

\begin{boxK}{Odds Ratio (OR)}

\begin{multicols}{2}

\[
\mathrm{OR}= \frac{\frac{a}{b}}{\frac{c}{d}}
= \frac{a \cdot d}{b \cdot c}
\]

\begin{tabular}{@{} r ll @{}}
\toprule
& Krankheit & Keine Krankheit \\
\midrule
Risikofaktor        & a & b \\
Ohne Risikofaktor   & c & d \\
\bottomrule
\end{tabular}

\end{multicols}
\end{boxK}

\begin{boxK}{Effektgrößen aus Effektgrößen}
\begin{multicols}{2}

\subsection*{Abstandsmaße aus Abstandsmaßen}
$$d=g\cdot\sqrt{\frac{n}{\mathrm{df}}}$$
$$g=d\cdot\sqrt{\frac{df}{n}}$$

\subsection*{Korrelationen aus Abstandsmaßen}
\begin{align*}
r&=\frac{d}{\sqrt{d^2+\frac{1}{p\cdot q}}}\\
r&=\sqrt{\frac{g^2(n_A\cdot n_B)}{g^2(n_A\cdot n_B)+(n_A+n_B)\mathrm{df}}}\\
\end{align*}
\subsection*{Abstandsmaße aus Korrelationen}
\begin{align*}
d&=\frac{r}{\sqrt{1-r^2}}\cdot\sqrt{\frac{1}{p\cdot q}}\\
g&=\frac{r}{\sqrt{1-r^2}}\cdot\sqrt{\frac{(n_A+n_B)\cdot df}{n_A\cdot n_B}}\\
\end{align*}

wobei $$p=\frac{n_A}{n_A+n_B}$$ und $$q=\frac{n_B}{n_A+n_B}$$

\end{multicols}
\end{boxK}

