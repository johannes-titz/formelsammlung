\begin{boxK}{Meta-Analyse}
  Mittlere, an der Stichprobe gewichtete, Effektgröße: $\bar{r} = \dfrac{\sum_{i=1}^n(N_i r_i)}{\sum_{i=1}^n N_i}$
    %\subsection{}
\end{boxK}

\begin{boxK}{Wahrscheinlichkeitstheorie}

\begin{multicols}{2}

\subsection*{Bedingte Wahrscheinlichkeit}

Die bedingte Wahrscheinlichkeit von $A$ unter der Bedingung $B$ wird notiert als $P(A | B)$ und berechnet sich durch:

\[
P(A | B) = \frac{P(A \cap B)}{P(B)}
\]

Hierbei ist $P(A \cap B)$ die Wahrscheinlichkeit, dass sowohl $A$ als auch $B$ eintreten.

\subsection*{Produktregel}

Die Produktregel beschreibt die Wahrscheinlichkeit des gemeinsamen Eintretens von zwei Ereignissen $A$ und $B$:

\[
P(A \cap B) = P(A | B) \cdot P(B)
\]

Alternativ kann die Produktregel auch in umgekehrter Reihenfolge geschrieben werden:

\[
P(A \cap B) = P(B | A) \cdot P(A)
\]

\subsection*{Summenregel}

Die Wahrscheinlichkeit, dass mindestens eines der Ereignisse $A$ oder $B$ eintritt:

\[
P(A \cup B) = P(A) + P(B) - P(A \cap B)
\]

Dies ist ein Spezialfall des allgemeinen Ein-/Ausschlussprinzips.

\subsection*{Wahrscheinlichkeitsrevision (Bayes)}
\[ P(A | B) = \frac{P(A) \cdot P(B | A) }{P(A) \cdot P(B | A) + P(\neg A) \cdot P(B | \neg A) } \]
\end{multicols}
\end{boxK}

