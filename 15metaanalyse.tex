\begin{boxK}{Meta-Analyse}
  Mittlere, an der Stichprobe gewichtete, Effektgröße: $\bar{r} = \dfrac{\sum_{i=1}^n(N_i r_i)}{\sum_{i=1}^n N_i}$
    %\subsection{}
\end{boxK}

\begin{boxK}{Wahrscheinlichkeitstheorie}

\begin{multicols}{2}

\subsection*{Bedingte Wahrscheinlichkeit}

Die bedingte Wahrscheinlichkeit von $A$ unter der Bedingung $B$ wird notiert als $P(A | B)$ und berechnet sich durch:

\[
P(A | B) = \frac{P(A \cap B)}{P(B)}
\]

Hierbei ist $P(A \cap B)$ die Wahrscheinlichkeit, dass sowohl $A$ als auch $B$ eintreten.

\subsection*{Additionssatz der Wahrscheinlichkeit}

Die Wahrscheinlichkeit, dass mindestens eines der Ereignisse $A$ oder $B$ eintritt, ergibt sich zu:

\[
P(A \cup B) = P(A) + P(B) - P(A \cap B)
\]

Sind $A$ und $B$ \emph{disjunkt} (d.\,h. $A \cap B = \emptyset$), so gilt die \emph{Summenregel}:
\[
P(A \cup B) = P(A) + P(B)
\]

\subsection*{Allgemeines Inklusions-Exklusions-Prinzip}

Für drei Ereignisse $A$, $B$ und $C$ gilt:

\[
\begin{aligned}
P(A \cup B \cup C)
&= P(A) + P(B) + P(C) \\
&\quad - [P(A \cap B) + P(A \cap C) + P(B \cap C)] \\
&\quad + P(A \cap B \cap C)
\end{aligned}
\]

Das Prinzip verallgemeinert sich für $n$ Ereignisse abwechselnd mit Addition und Subtraktion der Schnittwahrscheinlichkeiten aller $k$-fachen Überlappungen.

\subsection*{Produktregel}

Aus der Definition der bedingten Wahrscheinlichkeit folgt:

\[
P(A \cap B) = P(A | B) \cdot P(B)
\]

Alternativ kann die Produktregel auch in umgekehrter Reihenfolge geschrieben werden:

\[
P(A \cap B) = P(B | A) \cdot P(A)
\]

\subsection*{Stochastische Unabhängigkeit}

Zwei Ereignisse $A$ und $B$ heißen \emph{stochastisch unabhängig}, wenn das Eintreten des einen keinen Einfluss auf die Wahrscheinlichkeit des anderen hat:

\[
P(A | B) = P(A)
\quad \text{bzw.} \quad
P(B | A) = P(B)
\]

Dann folgt unmittelbar:

\[
P(A \cap B) = P(A) \cdot P(B)
\]

\subsection*{Satz der totalen Wahrscheinlichkeit}

Sei $\{B_1, B_2, \dots, B_n\}$ eine Zerlegung des Ergebnisraums in paarweise disjunkte Ereignisse mit $P(B_i) > 0$. Dann gilt:

\[
P(A) = \sum_{i=1}^n P(A | B_i) \cdot P(B_i)
\]

\subsection*{Wahrscheinlichkeitsrevision (Bayes)}

\[
P(A | B) = \frac{P(A) \cdot P(B | A)}{P(A) \cdot P(B | A) + P(A^{\mathrm{c}}) \cdot P(B | A^{\mathrm{c}})}
\]

\end{multicols}
\end{boxK}

