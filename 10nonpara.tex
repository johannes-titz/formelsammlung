
\begin{boxK}{Nonparametrische Verfahren}
%\section{Nonparametrische Verfahren}
\begin{multicols}{2}

\subsection*{$\chi^2$-Anpassungstests}
für eine Variable
\begin{align*}
\chi^2&=\sum\limits_i^k \frac{(f_{b,i}-f_{e,i})^2}{f_{e,i}}\\
f_{e,i}&=N \cdot P_i\\
\mathrm{df}&=k-1\\
\end{align*}
für zwei Variablen
\begin{align*}
\chi^2&=\sum\limits_{i}^k \sum\limits_{j}^m \frac{(f_{b,ij}-f_{e,ij})^2}{f_{e,ij}}\\
f_{e,ij}&=N \cdot P_{ij}\\
\mathrm{df}&=k \cdot m-1\\
\end{align*}

\subsection*{$\chi^2$-Unabhängigkeitstest}
\begin{align*}
\chi^2&=\sum\limits_{i}^k \sum\limits_{j}^m \frac{(f_{b,ij}-f_{e,ij})^2}{f_{e,ij}}\\
f_{e,ij}&=\frac{Z_i \cdot S_j}{N}\\
\mathrm{df}&=(k-1)\cdot(m-1)\\
\end{align*}
Bei 2 dichotomen Variablen: $\chi^2=\phi^2 \cdot N$\\

\columnbreak
\subsection*{U-Test nach Mann und Whitney (Wilcoxon Rangsummen-Test)}
\begin{enumerate}
    \item Messwerte insgesamt in Rangreihe bringen und jedem Messwert einen Rangplatz zuweisen
    \item T$_1$: Summe der Rangplätze der Gruppe 1
            T$_2$: Summe der Rangplätze der Gruppe 2
    \item $U=n_1\cdot n_2+\frac{n_1\cdot(n_1+1)}{2}-T_1$ \\
            $U^{\prime}=n_1 \cdot n_2+\frac{n_2 \cdot(n_2+1)}{2}-T2 = n_1 \cdot n_2-U$
    \item Prüfgröße für die bei uns verwendete Tabelle ist der kleinere der beiden U-Werte
\end{enumerate}



\subsection*{Wilcoxon-Test für AS (Vorzeichenrangtest)}
\begin{enumerate}
    \item Differenzen der Messwertpaare bilden
    \item den Beträgen der Differenzen Rangplatz zuweisen (beim kleinsten Betrag mit Rangplatz 1 beginnen)
    \item $T_-$: Summe der Rangplätze der negativen Differenzen
    $T_+$: Summe der Rangplätze der positiven Differenzen
    \item Prüfgröße für die Tabelle ist der kleinere der beiden T-Werte
\end{enumerate}

\subsection*{Notation}
\begin{tabular}{rl}
    $f_{b,i}$:&beobachtete Häufigkeit\\
    $f_{e,i}$:&erwartete Häufigkeit \\
    $P_i$: &in Merkmalsausprägung erwarteter Anteil\\
    $N$: &Anzahl der Untersuchungsteilnehmer\\
    $k$ ; $m$: &Anzahl der Merkmalsausprägungen beider\\ 
    &Merkmale\\
    $Z_i$: &Zeilenhäufigkeit \\
    $S_j$: &Spaltenhäufigkeit \\
    $\phi$: &Phi-Koeffizient\\
 \end{tabular}
 
\end{multicols}
\end{boxK}

